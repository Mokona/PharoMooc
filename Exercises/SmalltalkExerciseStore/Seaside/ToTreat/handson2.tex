\documentclass[a4paper,10pt]{scrartcl}
\usepackage[latin1]{inputenc}
\usepackage{graphicx}
\usepackage{alltt}
% for readable pdf files
\usepackage{times} 
\usepackage{graphicx}

\pagestyle{plain}
\voffset -5mm

\oddsidemargin  0mm
\evensidemargin -11mm
\marginparwidth 2cm
\marginparsep 0pt

\topmargin 0mm
\headheight 0pt
\headsep 0pt
\topskip 0pt        

%\footheight 0pt
%\footskip 0pt
\textheight 255mm
\textwidth 165mm

% ----------------------------------------
%
% II. Macros
%
% ----------------------------------------

\let\gaga=\phi \let\phi=\varphi \let\varphi=\gaga
\let\gaga=\theta \let\theta=\vartheta \let\vartheta=\gaga
\let\gaga=\epsilon \let\epsilon=\varepsilon \let\varepsilon=\gaga

\newcommand\code[1]{ {\footnotesize \texttt{#1}} }

\newcommand\nat{{\Bbb N}}
\def\int{{\Bbb Z}}
\newcommand\rat{{\Bbb Q}}
\newcommand\real{{\Bbb R}}
\newcommand\calD{{\cal D}}
\newcommand\calM{{\cal M}}
\newcommand\calL{{\cal L}}
\newcommand\Gone{\bfmath{G1}}%
\newcommand\Gonei{\bfmath{G1i}}%
\newcommand\Goneai{\bfmath{G1(i)}}%
\newcommand\Gtwo{\bfmath{G2}}%
\newcommand\Gtwoi{\bfmath{G2i}}%
\newcommand\Gtwoai{\bfmath{G2(i)}}%
\newcommand\Gthree{\bfmath{G3}}%
\newcommand\Gthreei{\bfmath{G3i}}%
\newcommand\Gthreeai{\bfmath{G3(i)}}%
\newcommand\True{\bfmath{True}}
\newcommand\False{\bfmath{False}}
\newcommand\FV{\itmath{FV}}
%
% mathrel's
%
\def\maps{\mathrel{:}}
\def\andthen{\mathrel{;}}
\def\suchthat{\mathrel{:}}
\def\defid{\mathrel{:\equiv}}
\def\defeq{\mathrel{:=}}
\def\eqdef{\mathrel{=:}}
\def\gewenn{\leftrightarrow}
\newcommand\tnt{\mathrel{\supset}}
\newcommand\pr{\mathrel{\vdash}}%
\def\prld#1#2{\mathrel{\vrule width 0.3pt height 8pt depth 2pt\mkern-2mu\textstyle{\hskip0.5ex\raise2pt\hbox{$\scriptstyle#1$}\hskip0.1ex\over\hskip0.5ex#2\hskip0.1ex}}}
\def\defGewenn{\mathrel{{:}{\Longleftrightarrow}}}
\def\Gewenn{\mathrel{\Longleftrightarrow}}
\def\To{\mathrel{\Longrightarrow}}
\def\defsimeq{\mathrel{:\simeq}}
\def\redone{\mathrel{\vartriangleright_1}}
\def\red{\mathrel{\vartriangleright}}
%
% mathbin's
%
\newlength{\lminus}
\settowidth{\lminus}{\hbox{$-$}}
\def\dotmin{\mathbin{\vbox{\lineskip0pt\baselineskip0pt\hbox to\lminus{\hfill.\hfill}\vglue-0.55ex\hbox{$-$}}}}

%
% mathop's
%
\def\pow{\mathop{\rm Pow}}
\def\dom{\mathop{\rm dom}}
\def\ran{\mathop{\rm ran}}
\def\seq#1{\langle #1\rangle}
\def\Seq{\mathop{\it Seq}}
\def\lh{\mathop{\it lh}}

% ----------------------------------------------------------------------
\newcommand{\vf}{\vspace{7mm plus7mm minus1mm}}
\newcommand{\compl}{{\rm c}}
\newcommand{\blank}{\hspace{0.5em}}
\newcommand\nyt{\rule{0pt}{0pt}}
\newcommand{\emspace}{\hspace{1em}}

\newcounter{countaufg}

\newenvironment{overview}[1]{%
      \bigskip%
      \subsection*{\selectfont #1%
	 \makeatother}%
      \bigskip%
      }      

\newenvironment{exercise}[1]{\addtocounter{countaufg}{1}%
      \bigskip%
      \subsection*{Exercise \arabic{countaufg} - %\series{m}
	   \selectfont #1%
	 \makeatother}%
      \bigskip%
      }      

\newenvironment{optexercise}[1]{\addtocounter{countaufg}{1}%
      \bigskip%
      \subsection*{Exercise \arabic{countaufg} (optional) - %\series{m}
	   \selectfont #1%
	 \makeatother}%
      \bigskip%
      }%

\newcommand\hint[1]{\emph{Hint :} #1}
\newcommand\note[1]{\emph{Note :} #1}

\newcommand\serieheader[6]{
\thispagestyle{empty}%
\begin{flushleft}
\includegraphics[scale=1]{../Framework/HeaderUSI}
\end{flushleft}
  \noindent%
  {\large\ignorespaces{\bf #1}\hspace{\fill}\ignorespaces{\bf #2}}\\ \\%
  {\large\ignorespaces #3 \hspace{\fill}\ignorespaces #4}\\
  \noindent%
  \hrule\par\bigskip\noindent%
  \bigskip\\ {\ignorespaces {\Large{\bf #5}} 
  \hspace{\fill}\ignorespaces \large {\Large{\bf #6}}}\\

  \hrule\par\bigskip\noindent%
}



\makeatletter
\def\enumerateMod{\ifnum \@enumdepth >3 \@toodeep\else
      \advance\@enumdepth \@ne
      \edef\@enumctr{enum\romannumeral\the\@enumdepth}\list
      {\csname label\@enumctr\endcsname}{\usecounter
        {\@enumctr}%%%? the following differs from "enumerate"
	\topsep0pt%
	\partopsep0pt%
	\itemsep0pt%
	%%%? end(differs)
	\def\makelabel##1{\hss\llap{##1}}}\fi}
\let\endenumerateMod =\endlist
\makeatother

\begin{document}

%%%%%%%%%%%%%%%%%%%%%%%%%%%%%%%%%%%%%%%%%%%%%%%%%%%%%%%%%%%%%%%%%%%%%

\serieheader{Programming Fundamentals 2 - Smalltalk}
	{2004 - 2005}
	{Prof. Michele Lanza}
	{Paolo Bonzini, Mircea Lungu and Romain Robbes}
	{Hands on 2 : collections, refactoring, streams}
	{16/03/2005 @ 10:30}


This hands-on session will tell you more about the collection classes.
They are a very powerful abstraction and you should learn to use their
methods.

It will also tell you about refactoring, which simply means reorganizing
code so that it has a better structure and can be reused more easily.
The VisualWorks environment includes very good tools for refactoring,
that help you keeping your code tidy.

Finally, you'll have a short introduction to streams, another powerful
abstraction in Smalltalk.  Basically streams provide a way to create
collections, especially strings like the \verb!printString!.  They are often
used and in your project you'll use something akin to a stream to
provide web content.

All these exercises will extend the \verb!FMPlayer! and \verb!FMPlayerPool!
classes, that you should get from Moodle.  Use
a Workspace so that you can easily look at the values of the variables
and inspect them.  In the Workspace, remember to refer to the classes
using the namespace, like \verb!ProFund.FMPlayer!.

\begin{exercise}{Collections}

To get more practice with collection, we will add more methods like
\verb!potentialAttackers!.

\begin{enumerate}
\item First, see if you grasped \verb!select:! and \verb!reject:!  Define
  a \verb!potentialGoalkeepers! method, which returns the players with a
  dexterity higher than or equal to 75.  Do not look at the code for
  \verb!potentialAttackers!: instead, compare the two methods after you are
  finished.

  Sooner or later, it will happen that all substitutions have been made and
  a red card is showed to the goalkeeper.  Surely you don't want to have a
  goalkeeper with a dexterity of 30, so we need a \verb!replacementGoalkeepers!
  method.  It should use \verb!reject:! to filter away players whose dexterity
  is lower than 30.

\item To show how \verb!collect:! is used, we will need a \verb!tshirtNumber!
  field in \verb!FMPlayer!.  Add it together with the getter,
  \verb!tshirtNumber!, and the setter, \verb!tshirtNumber:!.

  Don't forget to automatically generate the T-shirt number of the players!

  Now, add a \verb!tshirtNumbers! method to \verb!FMPlayerPool! that collects
  all the T-shirt numbers and return them.  \verb!collect:! is like
  \verb!map! in Scheme: it takes a one-argument block, and passes to the block
  all the objects in the collection, one at a time.  Then it returns a new
  collection, made from the values that the block returned.

\item There are various kinds of collection.  \verb!Set! is a collection
  that stores unique values.  You can convert a collection to a set with
  the \verb!asSet! method that the system provides to you: the resulting
  object will have duplicates removed.

  Building a \verb!Set! is often done to check if all the elements in a
  collection are unique.  How can you do it?

    \hint{If duplicates are removed, the \texttt{Set}'s \texttt{size} will be
      lower than the original collection's\ldots}

  Use this trick to create a \verb!numbersOkForGame! method in 
  \verb!FMPlayerPool!, that checks if the player in the pool have unique
  numbers.

\item There are also sorted collections.  For now, we will use them on
  collections of number or strings---in general, objects that respond to
  messages like \verb!<!, \verb!<=! etc.  If you have a collection of
  such objects, you can easily sort them using \verb!asSortedCollection!.

    \hint{Puzzled?  Try using \texttt{asSortedCollection} on an array of
      numbers in a workspace.  See what happens if you try to sort together
      numbers and strings.}

  Now, add a \verb!numbersForGame! method.  It will return the T-shirt
  numbers in sorted order if they are unique.  Otherwise it will answer nil.

\item Finally, you may want to add methods such as \verb!select:!, or
  \verb!collect:!, to the \verb!FMPlayerPool! directly!  This way, the
  class will behave more like a collection.

  These methods will not do anything interesting: they will simply
  \emph{delegate} their work to the \verb!players! instance variable.
  You may want \verb!at:! and \verb!size!, for example.
  Try adding a \verb!collect:! method to \verb!FMPlayerPool!, and
  use it in \verb!tshirtNumbers!.

% What about adding select: and having it return an FMPlayerPool?

\end{enumerate}
\end{exercise}

\begin{exercise}{Refactoring}

As you saw in the last part of exercise 1, changes are seldom local
to a single method.  For example, after adding \verb!collect:! to the
player pool class, it was natural to use it elsewhere.

\begin{enumerate}
\item One of the most useful refactorings in VisualWorks is "Extract
  method".  It takes a part of a method, and moves it to a newly created
  method.

  A natural place to apply this refactoring is \verb!generateRandomPlayers!.
  The whole body of the loop can be extracted to a new method which we'll
  call \verb!addRandomPlayer!.  Select with the mouse the body of the loop;
  in the code pane's popup menu, pick "Refactor" and then "Extract method".

\item Apply this refactoring again on the last line of
  \verb!addRandomPlayer!.  We'll call the resulting method \verb!addPlayer:!.
  VisualWorks finds out automatically that this method needs a parameter (the
  player it will put in the pool).

\item A more complex refactoring is "Extract method to component".  Follow
  these steps one by one to refactor \verb!potentialAttackers!, then do it
  again on \verb!potentialGoalkeepers!.

  The goal is to move \verb!each attack > 75! into a method of \verb!FMPlayer!,
  called \verb!isPotentialAttacker!.  The steps are:
  \begin{enumerate}
  \item select \verb!each attack > 75!.  Pick "Extract method to component".
  \item the player is in \verb!each!, so you want to move the method into
    \verb!each!.
  \item VisualWorks looks at the source code for your class and figures out
    that \verb!each! must be an \verb!FMPlayer!.  Still, it asks you for
    confirmation: press Ok.
  \item Finally, VisualWorks asks you for the name of the method it will
    create.  Type \verb!isPotentialAttacker!.
  \end{enumerate}

\end{enumerate}

Most other refactorings involve classes in an inheritance hierarchy.  They
are very useful, because they do work on many classes at once.  You will
learn more about them by practicing---all refactorings are found in the
popup menus of the browser, and most are enabled only once you select a
piece of code or an instance variable name.

\end{exercise}

\begin{exercise}{Streams}

Before we practice Streams, let's take a step back and do one more
exercise on collections.

Add a \verb!preferredRole! method to the \verb!FMPlayer! class.  First
it will pick the highest statistic among attack, defense and dexterity.
If it is lower than 60, the method will return \verb!#midfield!.  Otherwise,
the result will be \verb!#attacker! if the highest statistic is attack;
it will be \verb!#defense! if the highest statistic is defense;
it will be \verb!#goalkeeper! if the highest statistic is dexterity.

Now use a method similar to \verb!tshirtNumbers! to collect the preferred
roles of the team.  Call it \verb!teamDistribution! and put it in
\verb!FMPlayerPool!.

We are only interested in the number of attackers, the number of goalkeepers
etc.  So, we are interested in the number of duplicates but we don't want any
order to be enforced on the items: in this case, what we want is a \verb!Bag!.
To have \verb!teamDistribution! return a \verb!Bag!, send \verb!asBag! to
the collection returned by \verb!collect:!.

If you try to use \verb!teamDistribution!, you'll see that the output is
not very tidy: Bags are most useful when you have thousands of duplicates
of the same object, so it does not make much sense to print every object
N times.  An output like

\begin{verbatim}
   Bag(#goalkeeper:2 #attacker:4 #midfield:3 #defense:2)
\end{verbatim}

\noindent
would be better.  The output is quite complex, so overriding \verb!printString!
would not be very practical; instead we'll override \verb!printOn:!, which
prints an object's representation on a Stream.

Go to the Bag class (you can find it with the Hierarchy pane) and type this
code:

\begin{verbatim}
printOn: aStream
    self class printOn: aStream.
    aStream nextPut: $(.
    self sortedCounts do: [:each |
        each value printOn: aStream.
        aStream nextPut: $:.
        each key printOn: aStream.
        aStream nextPutAll: ' ' ].
    aStream nextPut: $)
\end{verbatim}

The method is quite complex, but you can see the basic methods:

\begin{itemize}
\item You can send \verb!nextPut:! to the Stream if you want to print
  a single Character on it.
\item You can send \verb!nextPutAll:! to the Stream if you want to append
  a whole String to it.  In this case I have used it to print a single
  space.
\item You can use the polymorphic \verb!printOn:! method to print other
  objects.
\end{itemize}

Actually, you should never override \verb!printString! like we have done
until now.  The correct way to change the printed representation of an object
is to override \verb!printOn:!, and the \verb!printString! will automatically
adjust.

\begin{enumerate}

\item Reimplement your \verb!printString! methods as \verb!printOn:!.
  Do not forget to remove your overridden \verb!printString! implementations.

\item Check that the inspector still prints your objects correctly.

\item Check that the \verb!players! variable of an \verb!FMPlayerPool!
  prints correctly.  In case you had not noticed, previously it printed
  just ``a ProFund.FMPlayer a ProFund.FMPlayer a ProFund.FMPlayer''.

\item How can the \verb!printString! automatically adjust to your
  \verb!printOn:! methods?  Look at how it is implemented in Object and
  try to figure this out.
\end{enumerate}
\end{exercise}
\end{document}
