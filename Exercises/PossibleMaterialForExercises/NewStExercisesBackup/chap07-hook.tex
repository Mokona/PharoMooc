\chapter{ Hook and Template Methods}



\section{Providing Hook Methods}



\paragraph{Current situation. }


The solution proposed for printing a Node displays the following 
information: 

\begin{code}
(Node withName: \#Node1 connectedTo: (Node new name: \#PC1)) printString

      Node named: Node1 connected to: PC1
\end{code}

A straightforward way to implement the printOn: method on the 
class Node is the following code:

\begin{code}
Node\texttt{>>}printOn: aStream

 aStream nextPutAll: 'Node named: ', self name asString.
 self hasNextNode 
  ifTrue:\ensuremath{[} aStream nextPutAll: ' connected to: ', self nextNode name\ensuremath{]}
\end{code}

However, with such implementation the printing of all kinds of 
nodes is the same.



\paragraph{New Requirements.}
To help in the understanding of the LAN we would like that depending 
on the specific class of node we obtain a specific printing like 
the following ones:

\begin{code}
 (Workstation withName: \#Mac connectedTo: (LanPrinter withName: 
\#PC1) printString

   Workstation Mac connected to Printer PC1

(LanPrinter withName: \#Pr1 connectedTo: (Node withName: \#N1) 
printString

 Printer Pr1 connected to Node N1
\end{code}

Define the method typeName that returns a string representing 
the name of the type of node in the `printing' protocol. This 
method should be defined in Node and all its subclasses. 

\begin{code}
(LanPrinter withName: \#PC1) typeName

      `Printer'

(Node withName: \#N1) typeName
     `Node'
\end{code}

 
Define the method simplePrintString on the class Node to provide 
more information about a node as show below:

\begin{code}
(Workstation withName: \#Mac connectedTo: (LanPrinter withName: 
\#PC1)) simplePrint

`Workstation Mac'

(LanPrinter withName: \#PC1) simplePrint

`Printer PC1'
\end{code}

Then modify the printOn: method of the class Node to produce 
the following output:
\begin{code}
(self withName: \#Mac connectedTo: (LanPrinter new name: 
\#PC1))

`Node Mac connected to Printer PC1'
\end{code}

\paragraph{Remark:} The method typeName is called a \textit{hook} method. This 
reflects the fact that it allows the subclasses to specialize 
the behavior of the superclass, here the printing of a all the 
different kinds of nodes. The method simplePrintString, even 
if in our case is rather simple, is called a template method. 
This name reflects the fact that the method specifies the context 
in which hook methods will be called and how they will fit into 
the template method to produce the expected result. 


Note that for abstract classes hook methods can be abstract too, 
one other case the hook method can propose a default behavior.

The Smalltalk class library contains a lot of such hooks that 
allows an easy customization of the proposed behavior. The proposed 
requirement already exists in the system. Study the method printOn: 
on the class Object. 

\endinput