\chapter{ Object Responsibility and Better Encapsulation}

\section{Reducing the coupling between classes }


To be a good citizen you as an object should follow as much as 
possible the following rules:

\begin{itemize}
\item
Be private. Never let somebody else play with your data.
\item
Be lazy. Let do other objects your job.
\item
Be focused. Do only one main task. 
\end{itemize}

While these guidelines are not really formal, one of the main 
consequences is that this is the responsibility of an object 
to provide a well defined interface protecting itself from its 
clients. The other consequence is that by delegating to other 
objects an object concentrates on a single task and responsibility. 
We now look how such guidelines can help us to provide better 
objects in our example.



\subsection{Current situation}


The interface of the packet class is really weak. It just provides 
free access to its data. The main impact of this weakness is 
the fact that the clients of the class Packet like Workstation relies 
on the internal coding of the Packet as shown in the first line 
of the following method.

\begin{code}
 Workstation\texttt{>>}accept: aPacket

    aPacket addressee = self name
       ifTrue:\ensuremath{[} Transcript show: 'A packet is accepted by the 
Workstation ', self name asString\ensuremath{]}
       ifFalse: \ensuremath{[}super accept: aPacket\ensuremath{]}
\end{code}


As a consequence, if the structure of the class Packet would 
change, the code of its clients would have to change too. Generalizing 
such a bad practice would lead to system that are badly coupled 
and being really difficult to change to meet new requirements.



\subsection{Solution.}

This is the responsibility of a packet to say if the packet is 
addressed to a particular node or if it was sent by a particular 
node.
\begin{itemize}
\item Define a method named isAddressedTo: aNode in `testing' protocol 
that answers if a given packet is addressed to the specified 
node. 
\item Define a method named isOriginatedFrom: aNode in `testing' protocol 
that answers if a given packet is originated from the specified 
node. 
\end{itemize}
Once these methods are defined, change the code of all the clients 
of the class Packet to call them.




\section{A Question of Creation Responsibility}


One of the problem with the previous approach for creating the 
nodes and the packets is the following:

it is the responsibility of the client of the objects to create 
them well-formed. For example, it is possible to create a node 
without specifying a name! This is a disaster for our LAN system 
(create an example method 3, and try it out). The same problem 
occurs with the packet: it is possible to create a packet without 
address nor contents.

We will find a solution to these problems.

\exercise  Define a class method named withName: in the class Node 
(protocol `instance creation') that creates a new node and assign 
its name.

\begin{code}
withName: aSymbol
....

\end{code}

Define a class method named withName:nextNode: in the class Node 
(protocol `instance creation') that creates a new node and assign 
its name and the next node in the LAN

\begin{code}
withName: aSymbol nextNode: aNode
....
\end{code}

Note that the first method can simply invoke the second one.

Define a class method named send:to\textit{:} in the class Packet (protocol `instance 
creation') that creates a new Packet with a contents and an address. 

\begin{code}
send: aString to: aSymbol\\
....
\end{code}

Now the problem is that we want to forbid the creation of non-well 
formed instances of these classes. To do so, we will simply redefine 
the creation method \textit{new} so that it will raise an error.

\exercise  Rewrite the new method of the class Node and Packet as 
the following:

\begin{code}
new

    self error: `you should invoke the method... to create a...' 
\end{code}

However, you have just introduced a problem: the instance creation 
methods you just wrote in exercise 11 will not work anymore, 
because they call \textit{new}, and that calling results in an error 
! The solution is to rewrite them such as 

\begin{code}
Node class\texttt{>>}withName: aSymbol nextNode: aNode
      ^ self basicNew initialize name: aSymbol ; nextNode: aNode
\end{code}

Do the same for the instance creation methods in class Packet.

\exercise Update and rerun your examples to make sure that 
your changes were correct.


Note that the previous code may break if a subclass specialize the nextNode: method
does not return the instance. To protect ourslef from possible unexpected
extension we add yourself that returns the receiver a the first cascaded message
(here name:), here the newly created instance. 

\begin{code}
Node class\texttt{>>}withName: aSymbol nextNode: aNode
      ^ self basicNew initialize name: aSymbol ; nextNode: aNode ; yourself
\end{code}


\section{Reducing the coupling between classes}


To be a good citizen you as an object should follow as much as 
possible the following rules:

\begin{itemize}
\item
Be private. Never let somebody else play with your private data.
\item
Be lazy. Let do other objects your job.
\item
Be focused. Do only one main task. 
\end{itemize}

While these guidelines are not really formal, one of the main 
consequences is that this is the responsibility of an object 
to provide a well defined interface protecting itself from its 
clients. The other consequence is that by delegating to other 
objects an object concentrates on a single task and responsibility. 
We now look how such guidelines can help us to provide better 
objects in our example.



\subsection{Current situation}

The interface of the packet class is really weak. It just provides 
free access to its data. The main impact of this weakness is 
the fact that the clients of the class Packet like Workstation relies 
on the internal coding of the Packet as shown in the first line 
of the following method.

\begin{code}
Workstation\texttt{>>}accept: aPacket

    aPacket addressee = self name\\
       ifTrue:\ensuremath{[} Transcript show: 'A packet is accepted by the Workstation ', self name asString\ensuremath{]}
       ifFalse: \ensuremath{[}super accept: aPacket\ensuremath{]}
\end{code}

As a consequence, if the structure of the class Packet would 
change, the code of its clients would have to change too. Generalizing 
such a bad practice would lead to system that are badly coupled 
and being really difficult to change to meet new requirements.



\subsection{Solution.}

This is the responsibility of a packet to say if the packet is 
addressed to a particular node or if it was sent by a particular 
node.

\begin{itemize}
\item Define a method named isAddressedTo: aNode in `testing' protocol 
that answers if a given packet is addressed to the specified 
node. 
\item Define a method named isOriginatedFrom: aNode in `testing' protocol 
that answers if a given packet is originated from the specified 
node. 
\end{itemize}


Once these methods are defined, change the code of all the clients 
of the class Packet to call them. You should note that a better 
interface encapsulates better the private data and the way they 
are represented. This allows one to locate the change in case 
of evolution.



\section{A Question of Creation Responsibility}


One of the problems with the first approach for creating the 
nodes and the packets is the following: it is the responsibility 
of the client of the objects to create them well-formed. For 
example, it is possible to create a node without specifying a 
name! This is a disaster for our LAN system, the node would never 
reachable, and worse the system would breaks because the assumptions 
that the name of a node is specified would not hold anymore (insert 
an anonymous node in Lan and try it out). The same problem occurs 
with the packet: it is possible to create a packet without address 
nor contents.

The solution to these problems is to give the responsibility 
to the objects to create well-formed instances. Several variations 
are possible:

\begin{itemize}
\item
When possible, providing default values for instance variable 
is a good way to provide well-defined instances.
\item
It is also a good solution to propose a consistent and well-defined 
creation interface. For example one can only provide an instance 
creation method that requires the mandatory value for the instance 
and forbid the creation of other instances.
\end{itemize}

\textbf{The class Packet.} 
We investigate the two solutions for the Packet class. For the 
first solution, the principle is that the creation method (new) 
should invoke an initialize method. Implement this solution. 
Just remember that new is sent to classes (a class method) and 
that initialize is sent to instances (instance method). Implement 
the method new in a `instance creation' protocol and initialize 
in a `initialize-release' protocol.

\begin{code}
Packet class\texttt{>>}new

\dots 

Packet\texttt{>>}initialize
   \dots 
\end{code}

The only default value that can have a default value is contents, 
choose 

\begin{code}
contents = `no contents'
\end{code} 


Ideally if each LAN would contain a default trash node, the default 
address and originator would point to it. We will implement this 
functionality in a future lesson. Implement first your own solution. 


\paragraph{Remarks and Analysis.} Note that with this solution it would 
be convenient to know if a packet contents is the default one 
or not. For this purpose you could provide the method hasDefaultContents 
that tests that. You can implement it in a clever way as shown 
below:

Instead of writing:

\begin{code}
Packet\texttt{>>}hasDefaultContents
 
 ^ contents = `no contents'

Packet\texttt{>>}initialize
 \dots 

contents := `no contents'
\dots 
\end{code}

You should apply the rule: `Say only once' and define a new method 
that returns the default content and use it as shown below:

\begin{code}
Packet\texttt{>>}defaultContents

    ^ `no contents'

Packet\texttt{>>}initialize
    \dots 

    contents := self defaultContent
    \dots 

Packet\texttt{>>}hasDefaultContent
    ^contents = self defaultContents
\end{code}

With this solution, we limit the knowledge to the internal coding 
of the default contents value to only one method. This way changing 
it does not affect the clients nor the other part of the class. 




\section{Proposing a creational interface}


\paragraph{Packet.}
We now apply the second approach by providing a better interface 
for creating packet. For this purpose we define a new creation 
method that requires a contents and an address.

Define a \textbf{class} methods named send:to\textit{:} and to: in the class Packet 
(protocol `instance creation') that creates a new Packet with a 
contents and an address. 

\begin{code}
Packet class\texttt{>>}send: aString to: aSymbol

....

Packet class\texttt{>>}to: aSymbol

....
\end{code}



\paragraph{The class Node.}

Now apply the same techniques to the class Node. Note that you 
already implemented a similar schema that the default value in 
the previous lessons. Indeed by default instance variable value 
is nil and you already implemented the method hasNextNode that 
to provide a good interface. \\
Define a \textbf{class} method named withName: in the class Node (protocol 
`instance creation') that creates a new node and assign its name.

\begin{code}
Node class\texttt{>>}withName: aSymbol

....
\end{code}

Define a \textbf{class} method named withName:connectedTo: in the class Node 
(protocol `instance creation') that creates a new node and assign 
its name and the next node in the LAN.

\begin{code}
 Node class\texttt{>>}withName: aSymbol connectedTo: aNode

....
\end{code}

Note that if to avoid to duplicate information, the first method 
can simply invoke the second one.



\section{Forbidding the Basic Instance Creation}


One the last question that should be discussed is the following 
one: should we or not let a client create an instance without 
using the constrained interface? There is no general answer, 
it really depends on what we want to express. Sometimes it could 
be convenient to create an uncompleted instance for debugging 
or user interface interaction purpose. 

Let us imagine that we want to ensure that no instance can be 
created without calling the methods we specified. We simply redefine 
the creation method new so that it will raise an error.

Rewrite the new method of the class Node and Packet as the following:

\begin{code}
Node class\texttt{>>}new

\tab \tab self error: `you should invoke the method... to create a...' 
\end{code}

However, you have just introduced a problem: the instance creation 
methods you just wrote in the previous exercise will not work 
anymore, because they call new, and that calling results in an 
error! Propose a solution to this problem.



\subsection{ Remarks and Analysis.}


A first solution could be the following code: 

\begin{code}
 Node class\texttt{>>}withName: aSymbol connectedTo: aNode

      ^ super new initialize name: aSymbol ; nextNode: aNode
\end{code}

However, even if the semantics permits such a call using super 
with a different method selector than the containing method one, 
it is a bad practice. In fact it implies an implicit dependency 
between two different methods in different classes, whereas the 
super normal use links two methods with the same name in two 
different classes. It is always a good practice to invoke the 
own methods of an object by using self. This conceptually avoids 
to link the class and its superclass and we can continue to consider 
the class as self contained.

The solution is to rewrite the method such as:

\begin{code}
Node class\texttt{>>}withName: aSymbol connectedTo: aNode

 ^  self basicNew initialize name: aSymbol ; nextNode: aNode
\end{code}

In Smalltalk there is a convention that all the methods starting 
with `basic' should not be overridden. basicNew is the method 
responsible for always providing an newly created instance. You 
can for example browse all the methods starting with `basic*' 
and limit yourself to Object and Behavior.

You can do the same for the instance creation methods in class Packet.



\section{Protecting yourself from your children}


The following code is a possible way to define an instance creation 
method for the class Node.

\begin{code}
Node class\texttt{>>}withName: aSymbol

 ^ self new name: aSymbol 
\end{code}

We create a new instance by invoking new, we assign the name 
of the node and then we return it. One possible problem with 
such a code is that a subclass of the class Node may redefine 
the method name: (for example to have a persistent object) and 
return another value than the receiver (here the newly created 
instance). In such a case invoking the method withName: on such 
a class would not return the new instance. One way to solve this 
problem is the following:


\begin{code}
Node class\texttt{>>}withName: aSymbol

 \ttt{|}newInstance\ttt{|}
 newInstance := self new.
 NewInstance name: aSymbol.
 ^ newInstance
\end{code}


This is a good solution but it is a bit too much verbose. It 
introduces extra complexity by the the extra temporary variable 
definition and assignment. A good Smalltalk solution for this 
problem is illustrated by the following code and relies on the 
use of the yourself message. 


\begin{code}
Node class\texttt{>>}withName: aSymbol

   ^ self new name: aSymbol ; yourself
\end{code}

yourself specifies that the receiver of the first message involved 
into the cascade (name: here and not new) is return. Guess what 
is the code of the yourself method is and check by looking in 
the library if your guess is right.





\endinput