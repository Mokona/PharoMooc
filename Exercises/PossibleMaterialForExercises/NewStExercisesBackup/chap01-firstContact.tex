\ifx\wholebook\relax\else
\documentclass{report}
\usepackage{times}
\usepackage{epsfig}
\usepackage{alltt}
\usepackage{xspace}
\usepackage{graphicx}
\usepackage{ifpdf}
\usepackage{ifthen}
\usepackage{amsmath}
\usepackage{a4wide}

\graphicspath{{figures/}} 

\ifpdf
\DeclareGraphicsExtensions{.pdf, .jpg, .tif, .png}
\else
\DeclareGraphicsExtensions{.eps, .jpg}
\fi

\newboolean{toseecomment}
\setboolean{toseecomment}{false}
%%change to false to hidde comment 
\newcommand{\comment}[1]{\ifthenelse{\boolean{toseecomment}}{$\blacktriangleright$ \textit{#1}$\blacktriangleleft$}{}}

\newcommand{\commented}[1]{}

\newboolean{seevwspecific}
\setboolean{seevwspecific}{true}
\newcommand{\vwspecific}[1]{\ifthenelse{\boolean{seevwspecific}}{#1}{}}

\newboolean{seecategoryspecific}
\setboolean{seecategoryspecific}{false}
\newcommand{\categoryspecific}[1]{\ifthenelse{\boolean{seecategoryspecific}}{#1}{}}

\newboolean{seestorespecific}
\setboolean{seestorespecific}{true}
\newcommand{\storespecific}[1]{\ifthenelse{\boolean{seestorespecific}}{#1}{}}

\newboolean{seesqueakspecific}
\setboolean{seesqueakspecific}{false}
\newcommand{\squeakspecific}[1]{\ifthenelse{\boolean{seesqueakspecific}}{#1}{}}


\newcommand{\category}[0]
{\ifthenelse{\boolean{seestorespecific}}
	{package\xspace}
	{category\xspace}}

\newcommand{\ct}[1]{\texttt{#1}\xspace}
\newcommand{\stc}[1]{{\small {\sf #1}}\xspace}
\newcommand{\ST}{{\textsc Smalltalk}\xspace}
\newcommand{\tab}{\makebox[4em]{}}
\newcommand{\ttt}[1]{{\tt #1}}
\newcommand{\chev}{\ttt{>>}}
\newcommand{\vw}{VisualWorks\xspace}
\newcommand{\sq}{Squeak\xspace}
\newcommand{\store}{Store\xspace}
\renewcommand{\chaptername}{Exercise}
\newcommand{\exercise}{\vspace{0.2cm}\noindent \textbf{Exercise:}\xspace}

\newsavebox{\fminibox}
\newlength{\fminilength}

% Fait un truc encadre
\newenvironment{fminipage}[1][\linewidth]
  {\setlength{\fminilength}{#1-2\fboxsep-2\fboxrule}
        \begin{lrbox}{\fminibox}\begin{minipage}{\fminilength}}
  { \end{minipage}\end{lrbox}\noindent\fbox{\usebox{\fminibox}}}

% Pareil mais pas encadre (a utiliser pour ne pas couper une fonction

\newenvironment{nminipage}[1][\linewidth]
  {\setlength{\fminilength}{#1}
        \begin{lrbox}{\fminibox}\begin{minipage}{\fminilength}}
  { \end{minipage}\end{lrbox}\noindent\mbox{\usebox{\fminibox}}}

% Un alltt encadre
\newenvironment{falltt}
  {\vspace*{0.3cm}\begin{fminipage}\begin{alltt}}
  {\end{alltt}\end{fminipage}\vspace*{0.3cm}}

% Un alltt pas encadre
\newenvironment{nalltt}
  {\vspace*{0.3cm}\begin{nminipage}\begin{alltt}}
  {\end{alltt}\end{nminipage}\vspace*{0.3cm}}

% Une fonction encadree
\newenvironment{ffonction}[1]
  {\begin{fonction}[#1]
        \begin{fminipage}
\begin{alltt}
\rule{\linewidth}{0.5pt}}
{\end{alltt}\end{fminipage}\end{fonction}}

\newenvironment{codeonepage}
  {\begin{nminipage}\vspace*{0.2cm}\hrule\vspace*{0.1cm}
\begin{alltt}}
  {\end{alltt} \vspace*{-0.2cm}\hrule \vspace*{0.2cm} \end{nminipage}}

\newenvironment{code}
  {\vspace*{0.1cm}\hrule\vspace*{-0.1cm}\begin{alltt}}
  {\end{alltt}\vspace*{-0.2cm}\hrule \vspace*{0.1cm}}



\begin{document}
\fi

\chapter{Basics of the VisualWorks Smalltalk Environment}

\section{Starting up}
Smalltalk is an interpreted language: the source code is translated 
to Smalltalk-byte codes, which is then interpreted and executed by 
the Smalltalk Virtual Machine. (Note that this is an approximation 
because Smalltalk dialects were also the first languages to develop 
Just in Time compilation, i.e. a method is compiled into byte-codes 
but also into native code that is directly called instead of  
interpreting the byte codes.)

When looking at VisualWorks Smalltalk, there are three important 
files:

\begin{description}
\item{visual.sou}  (ASCII): contains the textual code of the initial 
classes of the system. 
\item{visual.im} (Binary): contains byte code of all the object of the 
system, the libraries and the modifications you made.
\item{visual.cha} (ASCII): contains all the modifications made in the 
image-file since this was created.
\end{description}

On MacIntosh, to open an image:
\begin{itemize}
\item Drag the file 'visual.im' on the virtual machine to start the image.
\item If  you want to start your own image, just double click on it or 
drag it over the virtual machine.
\end{itemize}

On Solaris: you should invoke the virtual machine passing it an image 
as parameter. For the first opening, execute the first script that per default 
uses the original image the script is installation dependent but 
should look like path/bin/visualworks path/image/visual.im
Then after you can specify your own image. 

After opening the image, and thus starting a Smalltalk session, you 
see two windows : the VisualWorks launcher (with menu, buttons and a 
transcript), and a Workspace window (the one containing text). You 
can minimize or close this last one, since we do not need it for the 
moment.

The launcher is the starting point for working with your environment 
and for the opening of all the programming tools that you might need. 
To begin, we will first create a fresh image.

\paragraph{Creating a fresh image.}
We are going to create a new image for this lesson. 
\begin{itemize}
\item	select �Save As...� in the file-menu
\item	when the system prompts you for the name for the new image, you 
type lesson.
\item	the image is saved in the image directory 
\item	Have a look at the Transcript, and note what it says
\end{itemize}

The \stc{Transcript} is the lower part of the Launcher, and gives you 
system messages, like the one you see right now. We will see later on 
how you can put your own messages there.

\paragraph{About the mouse.}
VisualWorks (or rather the older precedent) was the first application 
to use multiple overlapping windows and a mouse. It extensively uses 
three mouse buttons, that are context sensitive and can be used 
everywhere throughout Smalltalk:
\begin{itemize}
\item	the left mouse button is the select button
\item	the middle button is the operate button
\item	the right button is the window button
\end{itemize}

On a Macintosh, where only one button is available, you have to use 
some keyboard keys together with pressing your mouse button:
\begin{itemize}
\item	the select button is the one button itself
\item	for the operate button, press the button while holding the alt-key 
pressed
\item	for the window button, press the button while holding the apple-key 
pressed
\end{itemize}

\section{Selecting text, and doing basic text manipulations}
One of the basic manipulations you do when programming is working 
with text. Therefore, this section introduces you to the different 
ways you can select text, and manipulate these selections.

The basic way of selecting text is by clicking in front of the first 
character you want to select, and dragging your mouse to the last 
character you want in the selection while keeping the button pressed 
down. Selected text will be highlighted. 

\exercise Select some parts of text in the Transcript.
You can also select a single word by double clicking on it. When the 
text is delimited by '' (single quotes), "" (double quotes), () 
(parentheses), [] (brackets), or {} (braces), you can select anything 
in between by double clicking just after the first delimiter.

\exercise Try these new selection techniques.


Now have a look at the text operations. Select a piece of text in the 
\stc{Transcript}, and bring on the operate menu. Note that you have to 
keep your mouse button pressed to keep seeing the window. 

\exercise Copy this piece of text, and paste it after your 
selection. Afterwards cut the newly inserted piece of text.

\exercise See if there is an occurrence of the word visual in the 
Transcript. Note that to find things in a text window, there is no 
need to select text. Just bring up the operate menu . 

\exercise Replace the word visual with C++ using the replace 
operation (if it does not contain Smalltalk, add this word or replace 
something else). Take your time and explore the different options of 
the replace operation.

\exercise Bring up the operate menu, but don't select anything yet. 
Press and hold the shift button, and select paste in the operation 
menu. What happens ?

\section{Opening a WorkSpace Window}
We will now open a workspace window, a text window much like the 
Transcript, you use to type text and expressions and evaluate them. 
To open a workspace:
\begin{itemize}
\item	select the tools menu in the Launcher
\item	from the tools menu, select Workspace
\item	You will see a framing rectangle (with your mouse in the upper left 
corner), that indicates the position where the Workspace will open. 
Before you click, you can move your mouse around to change this 
position. Click one time once you have found a good spot for your 
Workspace. 

\item	Now your mouse is in the bottom right corner, and you can adjust 
the size. If you click once more, once you have given it the size you 
like, the Workspace window appears.
\end{itemize}

This is the basic way of opening any kind of VisualWorks application 
windows. Experiment with it until you feel comfortable with it.

\paragraph{The Window menu.}
To resize a window on the Macintosh, click in the lower right corner 
while holding the alt-button. On a PC or Sun, you resize VisualWorks 
windows the same way as any other window.

Once you have opened your Workspace window, bring up the window menu, 
and experiment with it. Note that this menu is the same for each 
window, and contains the very basic window manipulations. 

\section{Evaluating Expressions}
In the Workspace, type : 3.
Select it, and bring up the operate menu.
In the operate you will see the next three different options for 
evaluating text and getting the result:

\begin{description}
\item[do it:] do it evaluates the current selection, and does not show any 
result of the evaluation result. 
\item[print it:] prints the result of the evaluation after your selection. 
The result is automatically highlighted, so you can easily delete it 
if you want to.
\item[inspect it:] opens an inspector on the result of the evaluation.
\end{description}

The distinction before these three operations is essential, so check 
that you REALLY understand their differences
\exercise Select 3, bring up the operate menu, and select print it.

\exercise Print the result of 3+4
\exercise Type \stc{Date today} and print it. Afterwards, select it again 
and inspect it.

After exercise 9, you will have an inspector on the result of the 
evaluation of the expression Date today (this tells VisualWorks to 
create an object containing the current date). This Inspector Window 
consists of two parts: the left one is a list view containing self (a 
pseudo variable containing the object you are inspecting) and the 
instance variables of the object. Right is a text field.

\exercise Click on \stc{self} in the inspector. What do you get ? Does 
it resemble the result shown by printstring ?

\exercise Select day. What do you get ? Now change this value, 
bring up the operate menu, and select accept it. Click again on 
\stc{self}. 
Any difference? 

\exercise In the inspector edit field, type the following: self 
weekday, select it and print it. This causes the message weekday to 
be sent to self (i.e. the date object), and the result is printed. 
Experiment with other expressions like:
\begin{code}
self daysInMonth
self monthName
\end{code}

Close the inspector when you are finished.
\exercise	Type in the Workspace the following expression: Time 
now, and inspect it. Have a look at self and the instance variables.

\exercise Type in the Workspace the following expression: \stc{Time 
dateAndTimeNow}. This tells VisualWorks to create an object 
representing both today's date and the current time, and open an 
inspector on it. Select the item \stc{self} in the inspector. [Note that 
self is an object called an Array. It holds on to two other objects 
(elements 1 and 2). You can inspect each element to get either the 
time or the date object.

\paragraph{Using the System Transcript.} 
We have already seen that the Transcript is a text window at the 
bottom of the Launcher where the system informs you important 
information. You can also use the Transcript yourself as a very cheap 
user interface.

If you have a Workspace open, place it so that it does not cover the 
System Transcript. Otherwise, open one and take care of  where you 
put it. Now, in the Workspace, type:
\begin{code}
	Transcript cr.
	Transcript show: 'This is a test'.
	Trancript cr.
\end{code}

Select these 3 lines and evaluate (do It) them with do it.
This will cause the string This is a test to be printed in the 
Transcript, preceeded and followed by a carriage return. Note that 
the argument of the show: message was a literal string (you see this 
because it is contained in single quotes). It is important to know, 
because the argument of the show: method always has to be a string. 
This means that if you want any non-string object to be printed (like 
a Number for example), you first have to convert it to a string by 
sending the message printString to it. For example, type in the 
workspace the following expression and evaluate it:

	\stc{Transcript show: 42 printString, 'is the answer to the Universe'}
Note here that the comma is used to concatenate the two strings that 
are passed to the show: message 42 printString and 'is the answer to 
the Universe'.

\exercise Experiment on your own with different expressions.
	\stc{Transcript cr ; show: �This is a test� ; cr}
Explain why this expression gives the same result that before. What 
is the semantics of  �;� ?



\ifx\wholebook\relax\else\end{document}\fi
