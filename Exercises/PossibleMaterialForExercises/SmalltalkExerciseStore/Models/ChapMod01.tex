\ifx\wholebook\relax\else
\documentclass{report}
\usepackage{times}
\usepackage{epsfig}
\usepackage{alltt}
\usepackage{xspace}
\usepackage{graphicx}
\usepackage{ifpdf}
\usepackage{ifthen}
\usepackage{amsmath}
\usepackage{a4wide}

\graphicspath{{figures/}} 

\ifpdf
\DeclareGraphicsExtensions{.pdf, .jpg, .tif, .png}
\else
\DeclareGraphicsExtensions{.eps, .jpg}
\fi

\newboolean{toseecomment}
\setboolean{toseecomment}{false}
%%change to false to hidde comment 
\newcommand{\comment}[1]{\ifthenelse{\boolean{toseecomment}}{$\blacktriangleright$ \textit{#1}$\blacktriangleleft$}{}}

\newcommand{\commented}[1]{}

\newboolean{seevwspecific}
\setboolean{seevwspecific}{true}
\newcommand{\vwspecific}[1]{\ifthenelse{\boolean{seevwspecific}}{#1}{}}

\newboolean{seecategoryspecific}
\setboolean{seecategoryspecific}{false}
\newcommand{\categoryspecific}[1]{\ifthenelse{\boolean{seecategoryspecific}}{#1}{}}

\newboolean{seestorespecific}
\setboolean{seestorespecific}{true}
\newcommand{\storespecific}[1]{\ifthenelse{\boolean{seestorespecific}}{#1}{}}

\newboolean{seesqueakspecific}
\setboolean{seesqueakspecific}{false}
\newcommand{\squeakspecific}[1]{\ifthenelse{\boolean{seesqueakspecific}}{#1}{}}


\newcommand{\category}[0]
{\ifthenelse{\boolean{seestorespecific}}
	{package\xspace}
	{category\xspace}}

\newcommand{\ct}[1]{\texttt{#1}\xspace}
\newcommand{\stc}[1]{{\small {\sf #1}}\xspace}
\newcommand{\ST}{{\textsc Smalltalk}\xspace}
\newcommand{\tab}{\makebox[4em]{}}
\newcommand{\ttt}[1]{{\tt #1}}
\newcommand{\chev}{\ttt{>>}}
\newcommand{\vw}{VisualWorks\xspace}
\newcommand{\sq}{Squeak\xspace}
\newcommand{\store}{Store\xspace}
\renewcommand{\chaptername}{Exercise}
\newcommand{\exercise}{\vspace{0.2cm}\noindent \textbf{Exercise:}\xspace}

\newsavebox{\fminibox}
\newlength{\fminilength}

% Fait un truc encadre
\newenvironment{fminipage}[1][\linewidth]
  {\setlength{\fminilength}{#1-2\fboxsep-2\fboxrule}
        \begin{lrbox}{\fminibox}\begin{minipage}{\fminilength}}
  { \end{minipage}\end{lrbox}\noindent\fbox{\usebox{\fminibox}}}

% Pareil mais pas encadre (a utiliser pour ne pas couper une fonction

\newenvironment{nminipage}[1][\linewidth]
  {\setlength{\fminilength}{#1}
        \begin{lrbox}{\fminibox}\begin{minipage}{\fminilength}}
  { \end{minipage}\end{lrbox}\noindent\mbox{\usebox{\fminibox}}}

% Un alltt encadre
\newenvironment{falltt}
  {\vspace*{0.3cm}\begin{fminipage}\begin{alltt}}
  {\end{alltt}\end{fminipage}\vspace*{0.3cm}}

% Un alltt pas encadre
\newenvironment{nalltt}
  {\vspace*{0.3cm}\begin{nminipage}\begin{alltt}}
  {\end{alltt}\end{nminipage}\vspace*{0.3cm}}

% Une fonction encadree
\newenvironment{ffonction}[1]
  {\begin{fonction}[#1]
        \begin{fminipage}
\begin{alltt}
\rule{\linewidth}{0.5pt}}
{\end{alltt}\end{fminipage}\end{fonction}}

\newenvironment{codeonepage}
  {\begin{nminipage}\vspace*{0.2cm}\hrule\vspace*{0.1cm}
\begin{alltt}}
  {\end{alltt} \vspace*{-0.2cm}\hrule \vspace*{0.2cm} \end{nminipage}}

\newenvironment{code}
  {\vspace*{0.1cm}\hrule\vspace*{-0.1cm}\begin{alltt}}
  {\end{alltt}\vspace*{-0.2cm}\hrule \vspace*{0.1cm}}


\begin{document}
\fi

\chapter{Modeling, modeling...}

\mainauthor{St\'ephane Ducasse}

\section{Classes or Instances}

\paragraph{Classes/Instances.} What are instances and what are classes in the following lists:

\begin{itemize}
\item Les bijoux de la castafiore, Comix, Asterix le gaulois, Book, Novel, Fahrenheit 451, Da Vinci Code.

\item Tokyo, Lisbon, Paris, Capital, Country, State, Madrid, France, Spain, Portugal

\item Video, Book, "le fabuleux destin d'amelie Poulain", Client, "la Bible", "Squeak", StarWars4, StarWars5, Dupont,
Dupond. GhostInTheShell, LeVoyageDeChihiro.

\end{itemize}

\section{Classification}

Classify the following abstractions: Opera, Film, Book, Novel, Comics.
What could be a possible common superclass?
To check if your hierarchy makes sense here is a guideline: It should be possible to substitute any instance of a superclasse by any instance of a subclass without breaking a program of the superclass.

\section{Capitals and countries}

\begin{itemize}
\item How do you relate the following entities: Lisbon, Paris, Capital, Country, State, Madrid, France, Spain, Portugal?
\item What are the possible relationships from countries to capitals and vice versa?
Write some examples.
\item How would you characterize countries and states in the context of international income?
\end{itemize}

\section{A Basic LAN Application}

Identify the classes and their possible responsibilities in the following example. Note that responsibilities are not state or properties of the objects but their behavior.

Imagine a local network simulator \textbf{Local Area Network (LAN)}. Workstations, nodes, printers are linked together and form a ring. Each node (workstations, nodes, printers) is pointing to a next node forming a ring. When a packet is received by a node it checks if this is for him, if this is the case, it performs appropriate actions (printing the contents of the packet for a printer), displaying it on a screen if this is a workstation). Only workstation can emit packet to reach other workstations or printers. 
When a workstation or a printer received a packet that is sent to it it does not forward it to the next node of the network.  A packet identifies who sent it, ist addressee and its contents. 




\section{Mail}
A mail has a title, one text, one sender, multiple recipients, and multiple attached files. What are the possible classes and their relationships.

\section{Hotel Reservation}

Monsieur Formulain, directeur d'une cha\^ine d'h\^otels, vous demande de concevoir une application de gestion pour ses h\^otels. Voici ce que vous devez mod\'eliser :

Un h\^otel Formulain est constitu\'e d'un certain nombre de chambres. Un responsable de l'h\^otel g\`ere la location des chambres. Chaque chambre se loue \`a un prix donn\'e (suivant ses prestations).

L'acc\`es aux salles de bain est compris dans le prix de la location d'une chambre. Certaines chambres comportent une salle de bain, mais pas toutes. Les h\^otes de chambres sans salle de bain peuvent utiliser une salle de bain sur le palier. Ces derni\`eres peuvent \^etre utilis\'ees par plusieurs h\^otes.

Les pi\`eces de l'h\^otel qui ne sont ni des chambres, ni des salles de bain (hall d'accueil, cuisine...) ne font pas partie de l'\'etude (hors sujet).

Des personnes peuvent louer une ou plusieurs chambres de l'h\^otel, afin d'y r\'esider. En d'autre termes : l'h\^otel h\'eberge un certain nombre de personnes, ses h\^otes (il s'agit des personnes qui louent au moins une chambre de l'h\^otel...).


\section{Classes/Instances.} 
Describe with the best accuracy the information that you would like to have to describe a video for an online catalog. You can get inspired by http://www.imdb.com/ or http://wrapper.rottentomatoes.com/. 

Now from the list StarWars4, StarWars5, DVDStarWarsBoite35, DVDStarWarsBoite35, Video, K7, LeVoyageDeChihiro, BoiteNoireLeVoyageDeChihiro145, BoiteNoireLeVoyageDeChihiro146, how would you identify instances and classes. 
For this item we suggest you to read http://students.engr.scu.edu/~gvenkata/typeobjectpattern.htm.

\section{Redo the Exercises using UML notation}

\ifx\wholebook\relax\else\end{document}\fi
