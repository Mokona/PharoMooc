%\documentstyle[11pt,epsf,french]{article}
%\topmargin -2.5cm
%\addtolength{\oddsidemargin}{-1.5cm}
%\addtolength{\evensidemargin}{-1.5cm}
%\addtolength{\textheight}{7cm}
%\addtolength{\textwidth}{4cm}



% $Author: ducasse $
% $Date: 2005/11/06 13:11:22 $
% $Revision: 1.2 $

\ifx\wholebook\relax\else
\documentclass{report}
\usepackage{times}
\usepackage{epsfig}
\usepackage{alltt}
\usepackage{xspace}
\usepackage{graphicx}
\usepackage{ifpdf}
\usepackage{ifthen}
\usepackage{amsmath}
\usepackage{a4wide}

\graphicspath{{figures/}} 

\ifpdf
\DeclareGraphicsExtensions{.pdf, .jpg, .tif, .png}
\else
\DeclareGraphicsExtensions{.eps, .jpg}
\fi

\newboolean{toseecomment}
\setboolean{toseecomment}{false}
%%change to false to hidde comment 
\newcommand{\comment}[1]{\ifthenelse{\boolean{toseecomment}}{$\blacktriangleright$ \textit{#1}$\blacktriangleleft$}{}}

\newcommand{\commented}[1]{}

\newboolean{seevwspecific}
\setboolean{seevwspecific}{true}
\newcommand{\vwspecific}[1]{\ifthenelse{\boolean{seevwspecific}}{#1}{}}

\newboolean{seecategoryspecific}
\setboolean{seecategoryspecific}{false}
\newcommand{\categoryspecific}[1]{\ifthenelse{\boolean{seecategoryspecific}}{#1}{}}

\newboolean{seestorespecific}
\setboolean{seestorespecific}{true}
\newcommand{\storespecific}[1]{\ifthenelse{\boolean{seestorespecific}}{#1}{}}

\newboolean{seesqueakspecific}
\setboolean{seesqueakspecific}{false}
\newcommand{\squeakspecific}[1]{\ifthenelse{\boolean{seesqueakspecific}}{#1}{}}


\newcommand{\category}[0]
{\ifthenelse{\boolean{seestorespecific}}
	{package\xspace}
	{category\xspace}}

\newcommand{\ct}[1]{\texttt{#1}\xspace}
\newcommand{\stc}[1]{{\small {\sf #1}}\xspace}
\newcommand{\ST}{{\textsc Smalltalk}\xspace}
\newcommand{\tab}{\makebox[4em]{}}
\newcommand{\ttt}[1]{{\tt #1}}
\newcommand{\chev}{\ttt{>>}}
\newcommand{\vw}{VisualWorks\xspace}
\newcommand{\sq}{Squeak\xspace}
\newcommand{\store}{Store\xspace}
\renewcommand{\chaptername}{Exercise}
\newcommand{\exercise}{\vspace{0.2cm}\noindent \textbf{Exercise:}\xspace}

\newsavebox{\fminibox}
\newlength{\fminilength}

% Fait un truc encadre
\newenvironment{fminipage}[1][\linewidth]
  {\setlength{\fminilength}{#1-2\fboxsep-2\fboxrule}
        \begin{lrbox}{\fminibox}\begin{minipage}{\fminilength}}
  { \end{minipage}\end{lrbox}\noindent\fbox{\usebox{\fminibox}}}

% Pareil mais pas encadre (a utiliser pour ne pas couper une fonction

\newenvironment{nminipage}[1][\linewidth]
  {\setlength{\fminilength}{#1}
        \begin{lrbox}{\fminibox}\begin{minipage}{\fminilength}}
  { \end{minipage}\end{lrbox}\noindent\mbox{\usebox{\fminibox}}}

% Un alltt encadre
\newenvironment{falltt}
  {\vspace*{0.3cm}\begin{fminipage}\begin{alltt}}
  {\end{alltt}\end{fminipage}\vspace*{0.3cm}}

% Un alltt pas encadre
\newenvironment{nalltt}
  {\vspace*{0.3cm}\begin{nminipage}\begin{alltt}}
  {\end{alltt}\end{nminipage}\vspace*{0.3cm}}

% Une fonction encadree
\newenvironment{ffonction}[1]
  {\begin{fonction}[#1]
        \begin{fminipage}
\begin{alltt}
\rule{\linewidth}{0.5pt}}
{\end{alltt}\end{fminipage}\end{fonction}}

\newenvironment{codeonepage}
  {\begin{nminipage}\vspace*{0.2cm}\hrule\vspace*{0.1cm}
\begin{alltt}}
  {\end{alltt} \vspace*{-0.2cm}\hrule \vspace*{0.2cm} \end{nminipage}}

\newenvironment{code}
  {\vspace*{0.1cm}\hrule\vspace*{-0.1cm}\begin{alltt}}
  {\end{alltt}\vspace*{-0.2cm}\hrule \vspace*{0.1cm}}


\begin{document}
\fi

\chapter{Les objets de Smalltalk-80}

%\newtheorem{exo}{Exercice}
%\title{{\bf DEUG A MIAS/SM/TI, 2$^{\grave eme}$ ann\'{e}e}\\
%TD-TP N$^0$ 1 : {\sc Les objets de Smalltalk-80}}
%\author{}
%\date{}
%\begin{document}
%\maketitle

\mainauthor{to be fixed: \pottier }
\metadata{VisualWorks}{?Squeak/VisualWorks?}{Universit\'e de Brest ---\pottier et al. }{?1.2?}{??}
\sd{fixer la version... peut-etre avoir des tests.}



%{\bf Points de rep\`ere sur le web}\\
%\begin{verbatim}
%http://penarvir.univ-brest.fr/deug98/
%\end{verbatim}

\section{Environnement Smalltalk-80}

\subsection{Notion de machine virtuelle}
Smalltalk est non seulement un langage de programmation mais aussi un syst\`eme d'exploitation, un environnement de d\'eveloppement et une biblioth\`eque de code tr\`es importante.\\
Comme Smalltalk est un interpreteur, il a besoin de conserver le texte initial (fichier {\it source}) et sa forme interne (fichier {\it image} - format byte code). Les modifications effectu\'ees sont enregistr\'ees dans un fichier {\it change}. Le repr\'esentation interne g\'en\'er\'ee peut \^etre vue comme un langage d'assemblage pour une machine virtuelle.

%\begin{figure}[h]
%\begin{center}
%\leavevmode
%\epsfxsize=250pt
%\epsfbox{../../cours97-98/DEUGOPT/virtuelST.eps}
%%\caption{le titre}
%\label{marge}
%\end{center}
%\end{figure}


\subsection{Acc\`es \`a l'image et au binaire}

L'image et le binaire se trouve~:
\begin{verbatim}
C:
   vwnc30
       vwnc30
           bin
           image
\end{verbatim}


Pour lancer une image, cliquer $2$ fois sur l'ic\^one de visualnc.im
dans le repertoire \verb|image|



\section{Organisation du travail}
Utilisation des boutons de la souris~:
\begin{description}
\item [bouton de gauche]~: s\'election de zone, activation de fen\^etre
\item [bouton de droite]~: pop-up menu (menu contextuel associ\'e \`a la fen\^etre
\item [bouton de droite dans la barre de titre]:  menu d'\'edition de la fen\^etre
\end{description}

Vous utiliserez le {\it FileList} se trouvant dans l'ic\^one bleue
repr\'esentant un tiroir pour cr\'eer des nouveaux fichiers
(cr\'eer a:TP1.txt)  dans le r\'epertoire Smalltalk.\\

Ecrire le texte du fichier (les exercices comment\'es) dans la fen\^etre du bas et sauvegarder en utilisant le menu contextuel (``Pop Up Menu''- bouton milieu de la souris- et {\it Save as}) de la fen\^etre.

Toute sauvegarde doit \^etre faite sur votre disquette, soit sous $a:$

\section{TP: Observation des objets et r\`egles de priorit\'e }
Evaluer~: selectionner une zone et faire ``print it''


Inspecter~: selectionner une zone et faire ``inspect''
\begin{itemize}
\item Inspecter les expressions suivantes~:
\begin{scode}
1
2.0
$a
'une chaine'
1@2
1.0@2.0
7/2
\end{scode}
\end{itemize}

Parmi les messages, on distingue 

\begin{description}
\item [ les messages unaires ] comme \ct{new}, \ct{sin}, \ct{sqrt}, \ct{size}, \ct{first}, \ct{last}, \ct{negated})
\item [ les messages binaires ] 
\verb?+ - * / ** // \\ < <= >  >= =  ~= == ~~  &  | @ ,?
\item [ les message \`a mot cl\'e ] comme \ct{at: put:}, \ct{x: y:}, \ct{bitOr:}, \ct{bitAnd:}
\end{description}
Dans une expression, on \'evalue en priorit\'e en respectant le parenth\'esage, les messages unaires puis binaires puis \`a mots cl\'es. Si l'expression ne comporte que des messages de m\^eme priorit\'e, l'\'evaluation se fait classiquement de la gauche vers la droite.

\begin{itemize}
\item Evaluer et inspecter les expressions suivantes~:
\begin{scode}
7.0/2.0
1 + 1
(1 + 1) printString
(1/2) class
\end{scode}
Expliquer pourquoi le parenth\'esage est obligatoire.
\item Pas de priorit\'e des op\'erateurs, l'\'evaluation suit l'ordre des messages. Evaluer~:
\begin{scode}
2 + 3*4
2 + (3*4)
2  + 1/2
2 + (1/2)
\end{scode}
\item Uniformit\'e des messages. Un m\^eme message peut \^etre adress\'e
\`a des objets de types diff\'erents. Evaluer et inspecter~:
\begin{scode}
2 sqrt
2.0 sqrt
(3/2) sqrt
(3/5) + (6/7)
\end{scode}
\item Arithm\'etique exacte et conversion de type. Evaluer~:
\begin{scode}
(11111111111111111111/11111111111111111112) + (1/11111111111111111112)
2r1000
16rFF
256 printStringRadix:2

(8 bitOr: 1) printStringRadix:2
(8 bitAnd: 9)
2e10 asInteger
\end{scode}
Faire le ou bit \`a bit des nombres binaires \ct{1010} et \ct{0011} et donner le r\'esultat en binaire.
\item Les variables. Evaluer~:
\begin{scode}
\stBar x \stBar
x := 2.

\stBar x \stBar
x := 2 + 1/2.

\stBar x \stBar
x := 2 + 1/2.
x := 2.
\end{scode}
Quelle est la valeur de \ct{x} apr\`es \'evaluation de la derni\`ere portion de code?
\item Le tableau est aussi un objet. Evaluer~:
\begin{scode}
\stBar table  \stBar 
table := #(1 2.0 'trois' 444444444444444444444444444444444444).
table := #(1 3 6 9).
table first.
table last.
table reverse.

\stBar string \stBar 
string := '#(1 3 6 9)'.
string first.
string last.
string reverse.
  
#(10 20 30 40) at: 2

\stBar table \stBar 
table := #(1 3 6 9).
table at:1 put: (3/4)

\stBar string \stBar 
string := '#(1 3 6 9)'.
string at: 4 put: $r
\end{scode}




\item Donner la liste des objets et des messages d\'efinis dans les expressions ci-dessous. Expliquer le r\'esultat de l'\'evaluation.



\begin{scode}
\stBar aPoint \stBar
aPoint:= Point x:2 y:1.
aPoint x: aPoint x * 2

\stBar x \stBar
x:=1.5.
x negated rounded.
Fraction  numerator: x*2 denominator: 3 + x negated rounded.
\end{scode}
\end{itemize}


\section{Exercices (A faire en TD)}
\subsection{Tableaux}

\begin{enumerate}

\item Multiplier  par 2 le 2 \`eme \'el\'ement d'un tableau,
\item Remplacer la valeur du 2 \`eme \'el\'ement d'un tableau par son oppos\'e.
\item  Remplacer la valeur du 3 \`eme  \'el\'ement par la valeur du 2 \`eme
\'el\'ement.
\item  Remplacer la valeur du 3 \`eme  \'el\'ement par la somme des 2 \`eme
et 3\`eme (ancienne valeur) \'el\'ements.
\item Le 2 \`eme du tableau \'etant une fraction, remplacer cette fraction par la fraction inverse dans le tableau.
\end{enumerate}

\subsection{Nombres}

\subsubsection{Maximum}

La m\'ethode \verb|max: unAutreNombre| appliqu\'e \`a un nombre
renvoie le plus grand des deux nombres.\\
Exemple~: \ct{2 max: 6} renvoie \ct{6}.

\begin{enumerate}
\item Calculer le maximum de 3 variables \ct{a b c} contenant des valeurs
quelconques.
\item Calculer le maximum de 3 variables \ct{a b c} contenant des valeurs
quelconques {\bf sans utiliser de variables interm\'ediaires}
\end{enumerate}

\subsubsection{Fonctions trigonom\'etriques}

Un nombre (en radians) comprend les messages correspondant aux 
fonctions trigonom\'etriques \ct{sin cos tan arcSin arcCos arcTan}.

On convertit un nombre de Degr\'e \`a Radian en lui envoyant
le message \ct{degreesToRadians}.

\paragraph{}
Calculer (\`a l'aide d'une fonction trigonom\'etrique) le c\^ot\'e
d'un carr\'e dont la diagonale mesure 1.41421 m\`etres.

\subsubsection{Conversion Celsius-Fahrenheit}

La formule de conversion Celsius-Fahrenheit est :
C = (5/9) (F - 32).\\
Convertir une variable contenant un nombre (en degr\'es Fahrenheit),
en degr\'es Celsius.

\subsubsection{Conversion binaire-hexad\'ecimale}

\verb|//| est l'op\'erateur de division enti\`ere.

\verb|\\| est l'op\'erateur de modulo.\\
Soit un nombre hexad\'ecimal (de 0 \`a 15 en base 10, de 0000 \`a 1111 en base 2),
on d\'esire le convertir en binaire
(sans utiliser \ct{printStringRadix:}).

\paragraph{}
En effectuant une s\'erie de divisions enti\`eres, les chiffres binaires
sont obtenus {\bf de la droite vers la gauche} gr\^ace au reste de la division
enti\`ere (le modulo).

\begin{enumerate}
\item Appliquer cet algorithme pour convertir le nombre hexa 15 en base 2.
\item V\'erifier en utilisant \ct{printStringRadix:}
\end{enumerate}
 
\subsection{Dates}

On obtient la date du jour gr\^ace au message \verb|today| envoy\'e
\`a la classe \verb|Date|.
\begin{scode}
\stBar d \stBar
d := Date today.
\end{scode}

On peut cr\'eer une date gr\^ace au message \ct{newDay:unNumeroJour monthNumber:unNumeroMois year:unNumeroAnnee} envoy\'e \`a la classe \ct{Date}.

On peut aussi cr\'eer une date gr\^ace au message \ct{fromDays:nombreJours} envoy\'e \`a la classe \ct{Date}.

La m\'ethode \ct{asDays} envoy\'e \`a une date renvoie le nombre de jours depuis le d\'ebut du si\`ecle (1/01/1901).

\begin{scode}
\stBar d \stBar
d := Date newDay:12 monthNumber:10 year:1998.
d asDays. "35713"
\end{scode}

\begin{enumerate}
\item Calculer le nombre de jours que vous avez d\'eja v\'ecu.
\item Calculer quelle serait la date, si vous aviez l'age que vous avez aujourd'hui
et si vous \'etiez n\'e le 1 Janvier 1901.
\end{enumerate}

\subsection{Caract\`eres}

On peut cr\'eer un caract\`ere \`a partir de son code ASCII en envoyant
le message \verb|value: unCodeAscii| \`a la classe \verb|Character|

On obtient le code ASCII d'un caract\`ere en lui envoyant le message
\ct{asInteger}

\begin{scode}
\stBar c \stBar
c := Character value: 65.
c asInteger. "65"
\end{scode}
 
\begin{enumerate}
\item Calculer le code ASCII de \verb|$a| et de \verb|$A|
\item Convertir un caract\`ere de minuscule \`a majuscule
\end{enumerate}

\subsection{Cha\^{\i}nes}

On obtient une cha\^{\i}ne \`a partir d'un caract\`ere en envoyant
le message \ct{with: unCaractere} \`a la classe \ct{String}.

Le caract\`ere "blanc" s'obtient en envoyant le message \ct{space} \`a la classe \ct{Character}.

On concat\`ene des cha\^{\i}nes avec l'op\'erateur \ct{,}

\begin{scode}
\stBar s1 s2 \stBar
s1 := String with: $1.
s2 := String with: $2.
s1, s2        "12"
\end{scode}

\begin{enumerate}
\item Fabriquer une cha\^{\i}ne contenant le caract\`ere "blanc"
\item Fabriquer une cha\^{\i}ne contenant votre pr\'enom, un caract\`ere
"blanc", votre nom
\end{enumerate}
 

\subsection{La classe \ct{Point}}
Dans la classe \ct{Point}, les messages unaires \ct{r} et \ct{theta} permettent de recup\'erer les coordonn\'ees polaires d'un point.

\begin{enumerate}
\item D\'efinir une translation de point par rapport \`a un vecteur donn\'e.
\item D\'efinir une homoth\'etie. 
\item Calculer l'angle form\'e par deux vecteurs.
\end{enumerate}



\ifx\wholebook\relax\else\end{document}\fi








