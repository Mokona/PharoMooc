%\documentstyle[11pt,epsf,french]{article}
%\topmargin -2.5cm
%\addtolength{\oddsidemargin}{-1.5cm}
%\addtolength{\evensidemargin}{-1.5cm}
%\addtolength{\textheight}{7cm}
%\addtolength{\textwidth}{4cm}



% $Author: ducasse $
% $Date: 2005/11/06 13:17:29 $
% $Revision: 1.1 $

\ifx\wholebook\relax\else
\documentclass{report}
\usepackage{times}
\usepackage{epsfig}
\usepackage{alltt}
\usepackage{xspace}
\usepackage{graphicx}
\usepackage{ifpdf}
\usepackage{ifthen}
\usepackage{amsmath}
\usepackage{a4wide}

\graphicspath{{figures/}} 

\ifpdf
\DeclareGraphicsExtensions{.pdf, .jpg, .tif, .png}
\else
\DeclareGraphicsExtensions{.eps, .jpg}
\fi

\newboolean{toseecomment}
\setboolean{toseecomment}{false}
%%change to false to hidde comment 
\newcommand{\comment}[1]{\ifthenelse{\boolean{toseecomment}}{$\blacktriangleright$ \textit{#1}$\blacktriangleleft$}{}}

\newcommand{\commented}[1]{}

\newboolean{seevwspecific}
\setboolean{seevwspecific}{true}
\newcommand{\vwspecific}[1]{\ifthenelse{\boolean{seevwspecific}}{#1}{}}

\newboolean{seecategoryspecific}
\setboolean{seecategoryspecific}{false}
\newcommand{\categoryspecific}[1]{\ifthenelse{\boolean{seecategoryspecific}}{#1}{}}

\newboolean{seestorespecific}
\setboolean{seestorespecific}{true}
\newcommand{\storespecific}[1]{\ifthenelse{\boolean{seestorespecific}}{#1}{}}

\newboolean{seesqueakspecific}
\setboolean{seesqueakspecific}{false}
\newcommand{\squeakspecific}[1]{\ifthenelse{\boolean{seesqueakspecific}}{#1}{}}


\newcommand{\category}[0]
{\ifthenelse{\boolean{seestorespecific}}
	{package\xspace}
	{category\xspace}}

\newcommand{\ct}[1]{\texttt{#1}\xspace}
\newcommand{\stc}[1]{{\small {\sf #1}}\xspace}
\newcommand{\ST}{{\textsc Smalltalk}\xspace}
\newcommand{\tab}{\makebox[4em]{}}
\newcommand{\ttt}[1]{{\tt #1}}
\newcommand{\chev}{\ttt{>>}}
\newcommand{\vw}{VisualWorks\xspace}
\newcommand{\sq}{Squeak\xspace}
\newcommand{\store}{Store\xspace}
\renewcommand{\chaptername}{Exercise}
\newcommand{\exercise}{\vspace{0.2cm}\noindent \textbf{Exercise:}\xspace}

\newsavebox{\fminibox}
\newlength{\fminilength}

% Fait un truc encadre
\newenvironment{fminipage}[1][\linewidth]
  {\setlength{\fminilength}{#1-2\fboxsep-2\fboxrule}
        \begin{lrbox}{\fminibox}\begin{minipage}{\fminilength}}
  { \end{minipage}\end{lrbox}\noindent\fbox{\usebox{\fminibox}}}

% Pareil mais pas encadre (a utiliser pour ne pas couper une fonction

\newenvironment{nminipage}[1][\linewidth]
  {\setlength{\fminilength}{#1}
        \begin{lrbox}{\fminibox}\begin{minipage}{\fminilength}}
  { \end{minipage}\end{lrbox}\noindent\mbox{\usebox{\fminibox}}}

% Un alltt encadre
\newenvironment{falltt}
  {\vspace*{0.3cm}\begin{fminipage}\begin{alltt}}
  {\end{alltt}\end{fminipage}\vspace*{0.3cm}}

% Un alltt pas encadre
\newenvironment{nalltt}
  {\vspace*{0.3cm}\begin{nminipage}\begin{alltt}}
  {\end{alltt}\end{nminipage}\vspace*{0.3cm}}

% Une fonction encadree
\newenvironment{ffonction}[1]
  {\begin{fonction}[#1]
        \begin{fminipage}
\begin{alltt}
\rule{\linewidth}{0.5pt}}
{\end{alltt}\end{fminipage}\end{fonction}}

\newenvironment{codeonepage}
  {\begin{nminipage}\vspace*{0.2cm}\hrule\vspace*{0.1cm}
\begin{alltt}}
  {\end{alltt} \vspace*{-0.2cm}\hrule \vspace*{0.2cm} \end{nminipage}}

\newenvironment{code}
  {\vspace*{0.1cm}\hrule\vspace*{-0.1cm}\begin{alltt}}
  {\end{alltt}\vspace*{-0.2cm}\hrule \vspace*{0.1cm}}


\begin{document}
\fi

\chapter{Les objets de Smalltalk-80}

%\newtheorem{exo}{Exercice}
%\title{{\bf DEUG A MIAS/SM/TI, 2$^{\grave eme}$ ann\'{e}e}\\
%TD-TP N$^0$ 1 : {\sc Les objets de Smalltalk-80}}
%\author{}
%\date{}
%\begin{document}
%\maketitle

\mainauthor{to be fixed: \pottier }
\metadata{VisualWorks}{?Squeak/VisualWorks?}{Universit\'e de Brest ---\pottier et al. }{?1.2?}{??}
\sd{fixer la version... peut-etre avoir des tests.}



\section{Observation des objets et r\`egles de priorit\'e }

\paragraph{Vocabulaire.} Nous utilisons le terme evaluer et inspecter pour deux actions distinctes. Evaluer~: selectionner une zone et faire ``print it''. 
Inspecter~: selectionner une zone et faire ``inspect''


\subsection{Inspecter les expressions suivantes}

\begin{scode}
1
2.0
$a
'une chaine'
1@2
1.0@2.0
7/2
\end{scode}


Parmi les messages, on distingue 

\begin{description}
\item [ les messages unaires ] comme \ct{new}, \ct{sin}, \ct{sqrt}, \ct{size}, \ct{first}, \ct{last}, \ct{negated})
\item [ les messages binaires ] 
\verb?+ - * / ** // \\ < <= >  >= =  ~= == ~~  &  | @ ,?
\item [ les message \`a mot cl\'e ] comme \ct{at: put:}, \ct{x: y:}, \ct{bitOr:}, \ct{bitAnd:}
\end{description}
Dans une expression, on \'evalue en priorit\'e en respectant le parenth\'esage, les messages unaires puis binaires puis \`a mots cl\'es. Si l'expression ne comporte que des messages de m\^eme priorit\'e, l'\'evaluation se fait classiquement de la gauche vers la droite.

\subsection{Evaluer et inspecter les expressions suivantes.}

\begin{scode}
7.0/2.0
1 + 1
(1 + 1) printString
(1/2) class
\end{scode}
Expliquer pourquoi le parenth\'esage est obligatoire dans les expressions pr\'ec\'edentes.

Attention, il n'y a pas de priorit\'e entre op\'erateurs, la m\'ethode \ct{+} est juste trait\'ee comme n'importe quelle autre m\'ethode. l'\'evaluation suit l'ordre des messages. 

Evaluer~:
\begin{scode}
2 + 3*4
2 + (3*4)
2  + 1/2
2 + (1/2)
\end{scode}

\subsection{Uniformit\'e des messages.} Un m\^eme message peut \^etre adress\'e
\`a des objets de types diff\'erents. Evaluer et inspecter~:

\begin{scode}
2 sqrt
2.0 sqrt
(3/2) sqrt
(3/5) + (6/7)
\end{scode}


\subsection{Arithm\'etique exacte et conversion de type.}

 Evaluer~:
\begin{scode}
(11111111111111111111/11111111111111111112) + (1/11111111111111111112)
2r1000
16rFF
256 printStringRadix:2

(8 bitOr: 1) printStringRadix:2
(8 bitAnd: 9)
2e10 asInteger
\end{scode}
Faire le ou bit \`a bit des nombres binaires \ct{1010} et \ct{0011} et donner le r\'esultat en binaire.


\subsection{Les variables.}

Evaluer~:
\begin{scode}
\stBar x \stBar
x := 2.

\stBar x \stBar
x := 2 + 1/2.

\stBar x \stBar
x := 2 + 1/2.
x := 2.
\end{scode}
Quelle est la valeur de \ct{x} apr\`es \'evaluation de la derni\`ere portion de code?

\subsection{Les tableaux sont aussi des objets.}

 Evaluer~:
\begin{scode}
\stBar table  \stBar 
table := #(1 2.0 'trois' 444444444444444444444444444444444444).
table := #(1 3 6 9).
table first.
table last.
table reverse.

\stBar string \stBar 
string := '#(1 3 6 9)'.
string first.
string last.
string reverse.
  
#(10 20 30 40) at: 2

\stBar table \stBar 
table := #(1 3 6 9).
table at:1 put: (3/4)

\stBar string \stBar 
string := '#(1 3 6 9)'.
string at: 4 put: $r
\end{scode}



\subsection{Liste des objets et des messages.}
Donner la liste des objets et des messages d\'efinis dans les expressions ci-dessous. Expliquer le r\'esultat de l'\'evaluation.



\begin{scode}
\stBar aPoint \stBar
aPoint:= Point x:2 y:1.
aPoint x: aPoint x * 2

\stBar x \stBar
x:=1.5.
x negated rounded.
Fraction  numerator: x*2 denominator: 3 + x negated rounded.
\end{scode}



\section{Tableaux}

\begin{enumerate}

\item Multiplier  par 2 le 2 \`eme \'el\'ement d'un tableau,
\item Remplacer la valeur du 2 \`eme \'el\'ement d'un tableau par son oppos\'e.
\item  Remplacer la valeur du 3 \`eme  \'el\'ement par la valeur du 2 \`eme
\'el\'ement.
\item  Remplacer la valeur du 3 \`eme  \'el\'ement par la somme des 2 \`eme
et 3\`eme (ancienne valeur) \'el\'ements.
\item Le 2 \`eme du tableau \'etant une fraction, remplacer cette fraction par la fraction inverse dans le tableau.
\end{enumerate}

\section{Nombres}

\subsection{Maximum}

La m\'ethode \verb|max: unAutreNombre| appliqu\'e \`a un nombre
renvoie le plus grand des deux nombres.\\
Exemple~: \ct{2 max: 6} renvoie \ct{6}.

\begin{enumerate}
\item Calculer le maximum de 3 variables \ct{a b c} contenant des valeurs
quelconques.
\item Calculer le maximum de 3 variables \ct{a b c} contenant des valeurs
quelconques {\bf sans utiliser de variables interm\'ediaires}
\end{enumerate}

\subsection{Conversion Celsius-Fahrenheit}

La formule de conversion Celsius-Fahrenheit est :
C = (5/9) (F - 32).\\
Convertir une variable contenant un nombre (en degr\'es Fahrenheit),
en degr\'es Celsius.


\ifx\wholebook\relax\else\end{document}\fi








