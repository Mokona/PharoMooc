% $Author: ducasse $
% $Date: 2005/05/16 13:38:12 $
% $Revision: 1.1.1.1 $

\ifx\wholebook\relax\else
\documentclass{report}
\usepackage{times}
\usepackage{epsfig}
\usepackage{alltt}
\usepackage{xspace}
\usepackage{graphicx}
\usepackage{ifpdf}
\usepackage{ifthen}
\usepackage{amsmath}
\usepackage{a4wide}

\graphicspath{{figures/}} 

\ifpdf
\DeclareGraphicsExtensions{.pdf, .jpg, .tif, .png}
\else
\DeclareGraphicsExtensions{.eps, .jpg}
\fi

\newboolean{toseecomment}
\setboolean{toseecomment}{false}
%%change to false to hidde comment 
\newcommand{\comment}[1]{\ifthenelse{\boolean{toseecomment}}{$\blacktriangleright$ \textit{#1}$\blacktriangleleft$}{}}

\newcommand{\commented}[1]{}

\newboolean{seevwspecific}
\setboolean{seevwspecific}{true}
\newcommand{\vwspecific}[1]{\ifthenelse{\boolean{seevwspecific}}{#1}{}}

\newboolean{seecategoryspecific}
\setboolean{seecategoryspecific}{false}
\newcommand{\categoryspecific}[1]{\ifthenelse{\boolean{seecategoryspecific}}{#1}{}}

\newboolean{seestorespecific}
\setboolean{seestorespecific}{true}
\newcommand{\storespecific}[1]{\ifthenelse{\boolean{seestorespecific}}{#1}{}}

\newboolean{seesqueakspecific}
\setboolean{seesqueakspecific}{false}
\newcommand{\squeakspecific}[1]{\ifthenelse{\boolean{seesqueakspecific}}{#1}{}}


\newcommand{\category}[0]
{\ifthenelse{\boolean{seestorespecific}}
	{package\xspace}
	{category\xspace}}

\newcommand{\ct}[1]{\texttt{#1}\xspace}
\newcommand{\stc}[1]{{\small {\sf #1}}\xspace}
\newcommand{\ST}{{\textsc Smalltalk}\xspace}
\newcommand{\tab}{\makebox[4em]{}}
\newcommand{\ttt}[1]{{\tt #1}}
\newcommand{\chev}{\ttt{>>}}
\newcommand{\vw}{VisualWorks\xspace}
\newcommand{\sq}{Squeak\xspace}
\newcommand{\store}{Store\xspace}
\renewcommand{\chaptername}{Exercise}
\newcommand{\exercise}{\vspace{0.2cm}\noindent \textbf{Exercise:}\xspace}

\newsavebox{\fminibox}
\newlength{\fminilength}

% Fait un truc encadre
\newenvironment{fminipage}[1][\linewidth]
  {\setlength{\fminilength}{#1-2\fboxsep-2\fboxrule}
        \begin{lrbox}{\fminibox}\begin{minipage}{\fminilength}}
  { \end{minipage}\end{lrbox}\noindent\fbox{\usebox{\fminibox}}}

% Pareil mais pas encadre (a utiliser pour ne pas couper une fonction

\newenvironment{nminipage}[1][\linewidth]
  {\setlength{\fminilength}{#1}
        \begin{lrbox}{\fminibox}\begin{minipage}{\fminilength}}
  { \end{minipage}\end{lrbox}\noindent\mbox{\usebox{\fminibox}}}

% Un alltt encadre
\newenvironment{falltt}
  {\vspace*{0.3cm}\begin{fminipage}\begin{alltt}}
  {\end{alltt}\end{fminipage}\vspace*{0.3cm}}

% Un alltt pas encadre
\newenvironment{nalltt}
  {\vspace*{0.3cm}\begin{nminipage}\begin{alltt}}
  {\end{alltt}\end{nminipage}\vspace*{0.3cm}}

% Une fonction encadree
\newenvironment{ffonction}[1]
  {\begin{fonction}[#1]
        \begin{fminipage}
\begin{alltt}
\rule{\linewidth}{0.5pt}}
{\end{alltt}\end{fminipage}\end{fonction}}

\newenvironment{codeonepage}
  {\begin{nminipage}\vspace*{0.2cm}\hrule\vspace*{0.1cm}
\begin{alltt}}
  {\end{alltt} \vspace*{-0.2cm}\hrule \vspace*{0.2cm} \end{nminipage}}

\newenvironment{code}
  {\vspace*{0.1cm}\hrule\vspace*{-0.1cm}\begin{alltt}}
  {\end{alltt}\vspace*{-0.2cm}\hrule \vspace*{0.1cm}}


\begin{document}
\fi

\chapter{Fundamentals on the Semantics of Self and Super}

\mainauthor{Ducasse, Wuyts}

This lesson wants you to give a better understanding of \ct{self} 
and \ct{super}. 

\section{self}

When the following message is evaluated: 

\begin{scode}
aWorkstation originate: aPacket
\end{scode}

The system starts to look up the method \ct{originate:} starts in 
the class of the message receiver: Workstation. Since this class 
defines a method \ct{originate:}, the method lookup stops and this 
method is executed.

Following is the code for this method:

\begin{scode}
Workstation\sep{}originate: aPacket

   aPacket originator: self.
   self send: aPacket
\end{scode}

\begin{enumerate}
    \item
It first sends the message \ct{originator:} to an instance of class \ct{Packet} with 
as argument self which is a pseudo-variable that represents the 
receiver of \ct{originate:} method. The same process occurs. The method \ct{originator:} 
is looked up into the class \ct{Packet}. As \ct{Packet} defines a method 
named \ct{originator:}, the method lookup stops and the method is executed. 
As shown below the body of this method is to assign the value 
of the first argument (aNode) to the instance variable \ct{originator}. 
Assignment is one of the few constructs of Smalltalk. It is not 
realized by a message sent but handle by the compiler. So no 
more message sends are performed for this part of \ct{originator:}.

\begin{scode}
Packet\sep{}originator: aNode

      originator := aNode
\end{scode}

\item
In the second line of the method \ct{originate:}, the message \ct{send: 
thePacket} is sent to \ct{self}. self represents the instance that receives 
the \ct{originate:} message. \textbf{The semantics of self specifies that 
the method lookup should start in the class of the message receiver.} 
Here \ct{Workstation}. Since there is no method \ct{send:} defined on the 
class \ct{Workstation}, the method lookup continues in the superclass 
of \ct{Workstation}: \ct{Node}. Node implements send\textit{:}, so the method 
lookup stops and send: is invoked

\end{enumerate}

\begin{scode}
 Node\sep{}send: thePacket

      self nextNode accept: thePacket
\end{scode}

The same process occurs for the expressions contained into the 
body of the method send:. 



\section{super}


Now we present the difference between the use of \ct{self} and \ct{super}. 
\ct{self} and \ct{super} are both pseudo-variables that are managed by 
the system (compiler). They both represents the receiver of the 
message being executed. However, there is no use to pass super 
as method argument, self is enough for this.

The main difference between self and super is their semantics 
regarding method lookup. 

\begin{itemize}
\item
The semantics of self is to start the method lookup \textbf{into 
the class of the message receiver and to continue in its superclasses.}\\
\item
The semantics of super is to start the method look into \textbf{the 
superclass of class} \textbf{in which the method being executed was 
defined and to continue in its superclasses.}. Take care the semantics 
is \textbf{NOT} to start the method lookup into the superclass of 
the receiver class, the system would loop with such a definition 
(see exercise 1 to be convinced). Using super to invoke a method 
allows one to invoke overridden method.
\end{itemize}

Let us illustrate with the following expression: the message accept: 
is sent to an instance of Workstation. 

\begin{scode}
aWorkstation accept: (Packet new addressee: \#Mac)
\end{scode}

As explained before the method is looked up into the class of 
the receiver, here Workstation. The method being defined into 
this class, the method lookup stops and the method is executed. 

\begin{scode}
 Workstation\sep{}accept: aPacket

     (aPacket addressee = self name)
         ifTrue: [ Transcript show: 'Packet accepted', self name asString ]
         ifFalse: [ super accept: aPacket ]
\end{scode}

Imagine that the test evaluates to false. The following expression 
is then evaluated.

\begin{scode}
 super accept: aPacket
\end{scode}


The method accept: is looked up in the superclass of the class 
in which the containing method accept: is defined. Here the containing 
method is defined into Workstation so the lookup starts in the 
superclass of Workstation: Node. The following code is executed 
following the rule explained before. 

\begin{scode}
Node\sep{}accept: aPacket

     self hasNextNode
         ifTrue: [ self send: aPacket ]
\end{scode}

\textbf{Remark.} The previous example does not show well the vicious 
point in the super semantics: the method look into \textbf{the superclass 
of class} \textbf{in whichor the method being executed was defined and 
not in the superclass of the receiver class.}

You have to do the following exercise to prove yourself that 
you understand well the nuance.


\exercise Imagine now that we define a subclass of Workstation 
called AnotherWorkstation and that this class does NOT defined 
a method accept:. Evaluate the following expression with both 
semantics:

\begin{scode}
 anAnotherWorkstation accept: (Packet new addressee: \#Mac)
\end{scode}

You should be convinced that the semantics of super change the 
lookup of the method so that the lookup (for the method via super) 
does NOT start in the superclass of the receiver class but in 
the superclass of the class in which the method containing the 
super. With the wrong semantics the system should loop.







\ifx\wholebook\relax\else\end{document}\fi

