% $Author: ducasse $
% $Date: 2005/05/16 13:38:14 $
% $Revision: 1.1.1.1 $

\ifx\wholebook\relax\else
\documentclass{report}
\usepackage{times}
\usepackage{epsfig}
\usepackage{alltt}
\usepackage{xspace}
\usepackage{graphicx}
\usepackage{ifpdf}
\usepackage{ifthen}
\usepackage{amsmath}
\usepackage{a4wide}

\graphicspath{{figures/}} 

\ifpdf
\DeclareGraphicsExtensions{.pdf, .jpg, .tif, .png}
\else
\DeclareGraphicsExtensions{.eps, .jpg}
\fi

\newboolean{toseecomment}
\setboolean{toseecomment}{false}
%%change to false to hidde comment 
\newcommand{\comment}[1]{\ifthenelse{\boolean{toseecomment}}{$\blacktriangleright$ \textit{#1}$\blacktriangleleft$}{}}

\newcommand{\commented}[1]{}

\newboolean{seevwspecific}
\setboolean{seevwspecific}{true}
\newcommand{\vwspecific}[1]{\ifthenelse{\boolean{seevwspecific}}{#1}{}}

\newboolean{seecategoryspecific}
\setboolean{seecategoryspecific}{false}
\newcommand{\categoryspecific}[1]{\ifthenelse{\boolean{seecategoryspecific}}{#1}{}}

\newboolean{seestorespecific}
\setboolean{seestorespecific}{true}
\newcommand{\storespecific}[1]{\ifthenelse{\boolean{seestorespecific}}{#1}{}}

\newboolean{seesqueakspecific}
\setboolean{seesqueakspecific}{false}
\newcommand{\squeakspecific}[1]{\ifthenelse{\boolean{seesqueakspecific}}{#1}{}}


\newcommand{\category}[0]
{\ifthenelse{\boolean{seestorespecific}}
	{package\xspace}
	{category\xspace}}

\newcommand{\ct}[1]{\texttt{#1}\xspace}
\newcommand{\stc}[1]{{\small {\sf #1}}\xspace}
\newcommand{\ST}{{\textsc Smalltalk}\xspace}
\newcommand{\tab}{\makebox[4em]{}}
\newcommand{\ttt}[1]{{\tt #1}}
\newcommand{\chev}{\ttt{>>}}
\newcommand{\vw}{VisualWorks\xspace}
\newcommand{\sq}{Squeak\xspace}
\newcommand{\store}{Store\xspace}
\renewcommand{\chaptername}{Exercise}
\newcommand{\exercise}{\vspace{0.2cm}\noindent \textbf{Exercise:}\xspace}

\newsavebox{\fminibox}
\newlength{\fminilength}

% Fait un truc encadre
\newenvironment{fminipage}[1][\linewidth]
  {\setlength{\fminilength}{#1-2\fboxsep-2\fboxrule}
        \begin{lrbox}{\fminibox}\begin{minipage}{\fminilength}}
  { \end{minipage}\end{lrbox}\noindent\fbox{\usebox{\fminibox}}}

% Pareil mais pas encadre (a utiliser pour ne pas couper une fonction

\newenvironment{nminipage}[1][\linewidth]
  {\setlength{\fminilength}{#1}
        \begin{lrbox}{\fminibox}\begin{minipage}{\fminilength}}
  { \end{minipage}\end{lrbox}\noindent\mbox{\usebox{\fminibox}}}

% Un alltt encadre
\newenvironment{falltt}
  {\vspace*{0.3cm}\begin{fminipage}\begin{alltt}}
  {\end{alltt}\end{fminipage}\vspace*{0.3cm}}

% Un alltt pas encadre
\newenvironment{nalltt}
  {\vspace*{0.3cm}\begin{nminipage}\begin{alltt}}
  {\end{alltt}\end{nminipage}\vspace*{0.3cm}}

% Une fonction encadree
\newenvironment{ffonction}[1]
  {\begin{fonction}[#1]
        \begin{fminipage}
\begin{alltt}
\rule{\linewidth}{0.5pt}}
{\end{alltt}\end{fminipage}\end{fonction}}

\newenvironment{codeonepage}
  {\begin{nminipage}\vspace*{0.2cm}\hrule\vspace*{0.1cm}
\begin{alltt}}
  {\end{alltt} \vspace*{-0.2cm}\hrule \vspace*{0.2cm} \end{nminipage}}

\newenvironment{code}
  {\vspace*{0.1cm}\hrule\vspace*{-0.1cm}\begin{alltt}}
  {\end{alltt}\vspace*{-0.2cm}\hrule \vspace*{0.1cm}}


\begin{document}
\fi

\chapter{ Extending the LAN Application}

This lesson uses the basic LAN-example and adds new classes and 
behaviour. Doing so, the design is extended to be more general 
and adaptive.


\mainauthor{Ducasse}
\sd{Extend with star and flesh a bit more the exercises}


\section{From a Ring to a Star}
Right now your lan is a ring, the token has to pass through the nodes one by one and this is not possible to send a packet to multiple nodes at once as this is the case in star architecture. 

\exercise Propose a solution to this problem. Hint having a new kind of node holding references to its connected node but acting as a normal node can be a good solution.




\section{Handling Loops}
When a packet is sent to an unknown node, it loops endlessly 
around the LAN. You will implement two solutions for this problem. 

\paragraph{Solution 1.} The first obvious solution is to avoid that a node 
resends a packet if it was the originator of the packet that 
it is sent. Modify the accept: method of the class Node to implement 
such a functionality. 

\paragraph{Solution 2.} The first solution is fragile because it relies on 
the fact that a packet is marked by its originator and that this 
node belongs to the LAN. A `bad' node could pollute the network 
by originate packets with a anonymous name. Think about different 
solutions.

Among the possible solutions, two are worth to be further analyzed:
\begin{enumerate}
\item 
Each node keeps track of the packets it already received. When 
a packet already received is asked to be accepted again by the 
node, the packet is not sent again in the LAN. This solution 
implies that packet can be uniquely identified. Their current 
representation does not allow that. We could imagine to tag the 
packet with a unique generated identifier. Moreover, each node 
would have to remember the identity of all the packets and there 
is no simple way to know when the identity of treated node can 
be removed from the nodes.
\item 
Each packet keeps track of the node it visited. Every time a 
packet aarrived at a node, it is asked if it has already been 
here. This solution implies a modification of the communication 
between the nodes and the packet: the node must ask the status 
of the packet. This solution allows the construction of different 
packet semantics (one could imagine that packets are broadcasted 
to all the nodes, or have to be accepted twice). Moreover once 
a packet is accepted, the references to the visited nodes are 
simply destroyed with the packet so there is no need to propagate 
this information among the nodes. 
\end{enumerate}

We propose you to implement the second solution so that the class \ct{Packet}
provides the following interface (the new responsibilities are 
in bold).

\begin{code}
Packet inherits from Object\\
Collaborators: Node\\
Responsibility:\\
addressee returns the addressee of the node to which the packet 
is sent.\\
contents describes the contents of the message sent.\\
originator references the node that sent the packet.\\
isAddressedTo: aNode answers if a given packet is addressed to 
the specified node. isOriginatedFrom: aNode answers if a given 
packet is originated from the specified node. \\
\textbf{isAcceptableBy: aNode} answers if a packet is acceptable by 
a node\\
\textbf{hasBeenAcceptedBy: aNode} tells a packet that it has been 
accepted by a given node. \\
\end{code}

\begin{itemize}
\item
New instance variable. A packet needs to keep track of the nodes 
it visited. Add a new instance variable called visitedNodes in 
the class \ct{Packet}. We want to collect the visited nodes in a set. Browse 
the class \ct{Set} and its superclass to find the function you need.\\
\item
Initialize the new instance variable. Modify the initialize methods 
of the class Packet so that the visitedNodes instance variable 
is initialized with an empty set. \\
\item
Node Acceptation Methods. In a protocol named `node acceptation', 
define the method \ct{isAcceptableBy:} and \ct{hasBeenAcceptedBy:}. \\
\item
Test if your implementation works by sending a `bad' node with 
a bad originator into the LAN. 
\end{itemize}



\section{Introducing a Shared Initialization Process}


As you noticed, each time a new class is created that is not 
a subclass of \ct{Node} we have to implement a new method whose the 
only purpose was to call the initialize method. We want to habe 
such a behavior specified only once and shared by all our Lan 
classes. 

Define a class \ct{LanObject} that inherits form \ct{Object}, implements 
an instance method initialize and a class method new that automatically 
calls the initialize method on the newly created object and retrun 
it. 

Then make all the classes that previously inherited from \ct{Object} 
inherit from \ct{LanObject} and check and remove if necessary if the 
unnecessary new methods. 



\section{Broadcasting and Multiple Addresses}
Up to now, when a packet reaches a node it is addressed to, the 
packet is handled by the node and the transmission of the packet 
is terminated (because is not sent to the next node in the network). 
In this exercise, we want you to provide facilities for broadcasting. 
If a node handles a packet that is broadcasted, the packet must 
be sent to the next node in the LAN instead of terminating the 
connection. For example, broadcasting makes it possible to save 
the contents of the same packet on different fileservers of the 
LAN. First try to solve this problem, and implement it afterwards.

In the current LAN, a packet only has one addressee. This exercise 
wants to add packets that have multiple addressees. Propose a 
solution for this problem, and implement it afterwards.



\section{Different Documents}
Suppose we have several kinds of documents (ASCII and Postscript) 
and two kinds of LANPrinter in the LAN (LANASCIIPrinter and LANPostscriptPrinter). 
We then want to make sure that every printer prints the right 
kind of document. Propose a solution for this problem.



\section{Logging Node}

We want to add a logging facility: this means each time a packet 
is sent from a node, we want to identify the node and the packet. 
Propose and implement a solution. Hint: introduce a new subclass 
of Node between Node and its subclasses and specialize the send: 
method. 

\section{Automatic Naming }


The name of a node have to be specified by its creator. We would 
like to have an automatic naming process that occurs when no 
name are specified. Note that the names should be unique. As 
a solution we propose you to use a counter, as this counter have 
to last over instance creations but still does not have any meaning 
for a particular node we use an instance variable of the class 
node.

Note that the NetworkManager could also be the perfect object 
to implement such a fonctionality.
We also would like that all the printer names start with Pr. 
Propose a solution. 

\stc{ Workstation Mac connected to Printer PC}

\ifx\wholebook\relax\else\end{document}\fi