\ifx\wholebook\relax\else
\documentclass{report}
\usepackage{times}
\usepackage{epsfig}
\usepackage{alltt}
\usepackage{xspace}
\usepackage{graphicx}
\usepackage{ifpdf}
\usepackage{ifthen}
\usepackage{amsmath}
\usepackage{a4wide}

\graphicspath{{figures/}} 

\ifpdf
\DeclareGraphicsExtensions{.pdf, .jpg, .tif, .png}
\else
\DeclareGraphicsExtensions{.eps, .jpg}
\fi

\newboolean{toseecomment}
\setboolean{toseecomment}{false}
%%change to false to hidde comment 
\newcommand{\comment}[1]{\ifthenelse{\boolean{toseecomment}}{$\blacktriangleright$ \textit{#1}$\blacktriangleleft$}{}}

\newcommand{\commented}[1]{}

\newboolean{seevwspecific}
\setboolean{seevwspecific}{true}
\newcommand{\vwspecific}[1]{\ifthenelse{\boolean{seevwspecific}}{#1}{}}

\newboolean{seecategoryspecific}
\setboolean{seecategoryspecific}{false}
\newcommand{\categoryspecific}[1]{\ifthenelse{\boolean{seecategoryspecific}}{#1}{}}

\newboolean{seestorespecific}
\setboolean{seestorespecific}{true}
\newcommand{\storespecific}[1]{\ifthenelse{\boolean{seestorespecific}}{#1}{}}

\newboolean{seesqueakspecific}
\setboolean{seesqueakspecific}{false}
\newcommand{\squeakspecific}[1]{\ifthenelse{\boolean{seesqueakspecific}}{#1}{}}


\newcommand{\category}[0]
{\ifthenelse{\boolean{seestorespecific}}
	{package\xspace}
	{category\xspace}}

\newcommand{\ct}[1]{\texttt{#1}\xspace}
\newcommand{\stc}[1]{{\small {\sf #1}}\xspace}
\newcommand{\ST}{{\textsc Smalltalk}\xspace}
\newcommand{\tab}{\makebox[4em]{}}
\newcommand{\ttt}[1]{{\tt #1}}
\newcommand{\chev}{\ttt{>>}}
\newcommand{\vw}{VisualWorks\xspace}
\newcommand{\sq}{Squeak\xspace}
\newcommand{\store}{Store\xspace}
\renewcommand{\chaptername}{Exercise}
\newcommand{\exercise}{\vspace{0.2cm}\noindent \textbf{Exercise:}\xspace}

\newsavebox{\fminibox}
\newlength{\fminilength}

% Fait un truc encadre
\newenvironment{fminipage}[1][\linewidth]
  {\setlength{\fminilength}{#1-2\fboxsep-2\fboxrule}
        \begin{lrbox}{\fminibox}\begin{minipage}{\fminilength}}
  { \end{minipage}\end{lrbox}\noindent\fbox{\usebox{\fminibox}}}

% Pareil mais pas encadre (a utiliser pour ne pas couper une fonction

\newenvironment{nminipage}[1][\linewidth]
  {\setlength{\fminilength}{#1}
        \begin{lrbox}{\fminibox}\begin{minipage}{\fminilength}}
  { \end{minipage}\end{lrbox}\noindent\mbox{\usebox{\fminibox}}}

% Un alltt encadre
\newenvironment{falltt}
  {\vspace*{0.3cm}\begin{fminipage}\begin{alltt}}
  {\end{alltt}\end{fminipage}\vspace*{0.3cm}}

% Un alltt pas encadre
\newenvironment{nalltt}
  {\vspace*{0.3cm}\begin{nminipage}\begin{alltt}}
  {\end{alltt}\end{nminipage}\vspace*{0.3cm}}

% Une fonction encadree
\newenvironment{ffonction}[1]
  {\begin{fonction}[#1]
        \begin{fminipage}
\begin{alltt}
\rule{\linewidth}{0.5pt}}
{\end{alltt}\end{fminipage}\end{fonction}}

\newenvironment{codeonepage}
  {\begin{nminipage}\vspace*{0.2cm}\hrule\vspace*{0.1cm}
\begin{alltt}}
  {\end{alltt} \vspace*{-0.2cm}\hrule \vspace*{0.2cm} \end{nminipage}}

\newenvironment{code}
  {\vspace*{0.1cm}\hrule\vspace*{-0.1cm}\begin{alltt}}
  {\end{alltt}\vspace*{-0.2cm}\hrule \vspace*{0.1cm}}


\begin{document}
\fi


\chapter{SmallWiki Introduction}

\sd{todo: have some wiki extensions from lukas master}
%%%%%%%%%%%%%%%%%%%%%%%%%%%%%%%%%%%%%%%%%%%%%%%
%% Starting Smallwiki
%%%%%%%%%%%%%%%%%%%%%%%%%%%%%%%%%%%%%%%%%%%%%%

\section{Start Smallwiki}
\exercise Start the SmallWiki-Server on your machine (use the SmallWiki workspace) and explore
all its possibilities from the user-point of view. If you are
asked to login, give 'admin' as username and 'smallwiki' as
password. Note: The authentication is still experimental and it is
known not to work with the Mac OS-X Web browser \emph{Safari}.

%%%%%%%%%%%%%%%%%%%%%%%%%%%%%%%%%%%%%%%%%%%%%%
%% Play with Glossary
%%%%%%%%%%%%%%%%%%%%%%%%%%%%%%%%%%%%%%%%%%%%%%
\section{Play with Glossary}


\exercise Load the package \textit{SmallWiki Glossary} from Store.
To do this, go to the \textbf{Store} menu in the
\textbf{VisualWorks} main window. Choose the \textbf{Published
Items} menu item. Once the window with all the published items in
Store appears, look for the package \textbf{SmallWiki Glossary} in
the left pane of the window (you might have to scroll past the \emph{bundles} that are listed first). Click in the package and now choose
the last version of the package that appears in the right pane of
the window. With the right-button of the mouse, choose the
\textbf{Load} option. Once the process is
completed, the package is loaded in your system.

Then go back to your browser and add a new
glossary-component to the root of your wiki. \comment {What is
this? They have to add a new glossary component once the package
is loaded}\\

\textit{SmallWiki Glossary}is a small component to create
keyword-indexes of a whole wiki. In that package there are only 3
classes. Explore them and try to answer the following questions:

\begin{enumerate}

\item Glossary: What is the supper-class of Glossary. Why is the
instance side empty? What is done on the class-side? What would
you have to do to implement a caching mechanism?
\comment{The
caching mechanism is not obvious--ab}

\item GlossaryView: What is the supper-class of GlossaryView? What is the responsiblity of GlossaryView? What piece of code would you have to change to reverse the sort-order of the keywords?

\item VisitorGlossary: From where is this visitor called? Why does the visitor override the message \ct{\#defaultCollection}. Why is the message \ct{\#acceptCode:} empty? Should this message be removed? What would you have to change, if you wanted to filter-out common words as \ct{the}, \ct{and}, \ct{or}, \ct{not} ?
\end{enumerate}


%%%%%%%%%%%%%%%%%%%%%%%%%%%%%%%%%%%%%%%%%%%%%%%%%%%%%%%%%%%%%%%%%%
%% SmallWiki Tests
%%%%%%%%%%%%%%%%%%%%%%%%%%%%%%%%%%%%%%%%%%%%%%%%%%%%%%%%%%%%%%%%%%
\section{Diving into SmallWiki Tests}

\exercise Go to the package \textit{SmallWiki Tests} in the
\textit{SmallWiki} bundle. Select all the test-classes and hit on
the run-button (assuming that \ct{RBSUnitExtensions} has been
loaded). Do you get errors?
 \comment {It is supposed that they already load SmallWikiTest packages?}

\begin{enumerate}
\item Now have a look at \ct{StructureTests}. First have a
look at \ct{StructureTests\sep{}setUp}\footnote{the notation
\ct{AClass\sep{}aMethodName} refers to the method named
\ct{aMethodName} defined in the class \ct{AClass}. Method
defined in a metaclass can be referenced such as \ct{AClass
class\sep{}aMethodName}} and try to understand how a wiki is built
from code.

\item Now go to the 'testing-testing' protocol. By looking at those tests only, guess what the messages \ct{Structure\sep{}isEmtpy} and \ct{Structure\sep{}isRoot} are used for.

\item Do the same in the 'testing-navigation' protocol for the
messages \ct{Structure\sep{}next} and \\ \ct{Structure\sep{}first}.
\end{enumerate}

%%%%%%%%%%%%%%%%%%%%%%%%%%%%%%%%%%%%%%%%%%%%%%%%%%%%%%%%%%%%%%%%%%
%% Structure
%%%%%%%%%%%%%%%%%%%%%%%%%%%%%%%%%%%%%%%%%%%%%%%%%%%%%%%%%%%%%%%%%%
\section{Understanding the Structure}
\exercise Now change to the package \textit{SmallWiki Structure} and select the class \ct{Structure}. What are its subclasses? Have you already seen those classes anywhere?
\begin{enumerate}
\item Have a look at the messages \ct{Structure\sep{}isEmpty},
\ct{Structure\sep{}isRoot}, \ct{Structure\sep{}next} and
\ct{Structure\sep{}first}. Do they do the same as you thought
while studying the tests? \item Now go to the \textit{serving}
protocol. Have a close look at all the messages defined in there.
What are they used for? What is the starting point?. Why is
\ct{Structure\sep{}processChild:} overridden by the Folder class?
What is its default implementation?
\end{enumerate}

\ifx\wholebook\relax\else\end{document}\fi
