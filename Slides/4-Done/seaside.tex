\documentclass[9pt,compress]{beamer}

\usepackage{verbatim}
\begin{comment}

0. distribuer one click image et intégrer le code dans le package cache
1. 


\end{comment}

\usepackage[utf8]{inputenc}
\usepackage[T1]{fontenc}

\usepackage{xcolor}
\usepackage{url}
\usepackage{sourcecode}
\usepackage{relsize}

%\usepackage{color}
%\usepackage{beamerthemeruby}
\usetheme{Madrid}
\usepackage{themeslides}
\usepackage{lmodern}
% \usepackage{foekfont}

\usepackage{mathptmx} % math font en times
% \usepackage{amsmath,amssymb,latexsym}

% \usepackage{pgf}
\graphicspath{{./imgs/}}

\usepackage{tikz}
% \usetikzlibrary{positioning} % cube
\usetikzlibrary{arrows,shapes}
\usetikzlibrary{trees} 

\usepackage[frenchb]{babel}

\usepackage{booktabs}

\usepackage[lined]{algorithm2e}

\beamertemplateballitem
% \beamersetleftmargin{0.3em} % réduit énormément les marges. 
% \beamersetrightmargin{0.3em} % Les blocks dépassent meme dans les marges


\mode<handout>{
  \usepackage{pgfpages}
  \pgfpagesuselayout{8 on 1}[a4paper,border shrink=5mm]
  \setbeamercolor{background canvas}{bg=black!5}
}

% Delete this, if you do not want the table of contents to pop up at
% the beginning of each subsection:

\AtBeginSection[]
{
  \begin{frame}<beamer>
    \frametitle{Plan}
    \small\tableofcontents[currentsection,subsectionstyle=show/shaded/hide]
  \end{frame}
}

\AtBeginSubsection[] % Do nothing for \subsection* 
{ 
\begin{frame}<beamer> 
\frametitle{Plan}
\small\tableofcontents[currentsection,currentsubsection,subsectionstyle=show/shaded/hide] 
\end{frame} 
} 


\title{Une introcution pragmatique à Seaside\\ 
  {\smaller --- Le framework en Smalltalk pour développer des applications Web sophistiquées ---}\vskip-0.5em}
\author{Luc Fabresse\\
  {\smaller luc.fabresse@mines-douai.fr}}
\institute{École des mines de Douai -- Option ISIC -- Dépt. I.A. }
\date{\today}
\titlegraphic{\includegraphics[width=0.6\paperheight]{cover}}


\newcommand{\tagcloud}[2]{
	\ifnum #1=-4 \tiny 			\color{mystructurecolor!20} 	\else
	\ifnum #1=-3 \scriptsize 	\color{mystructurecolor!35}	\else
	\ifnum #1=-2 \footnotesize	\color{mystructurecolor!45}	\else
	\ifnum #1=-1 \small			\color{mystructurecolor!55}	\else
	\ifnum #1=0  \normalsize 	\color{mystructurecolor!65}	\else
	\ifnum #1=1  \large			\color{mystructurecolor!85}	\else
	\ifnum #1=2  \Large			\color{mystructurecolor!95}	\else
	\ifnum #1=3  \LARGE			\color{black!05!mystructurecolor}	\else
	\ifnum #1=4  \huge			\color{black!15!mystructurecolor}	\else
	\ifnum #1=5  \Huge			\color{black!45!mystructurecolor}	\fi
	\fi	\fi \fi \fi \fi \fi	\fi	\fi	\fi
	\textbf{#2}	
}

\newcommand{\code}[1]{\texttt{\textup{\NoAutoSpaceBeforeFDP #1\AutoSpaceBeforeFDP}}} %\xspace

\begin{document}

% \frame[plain]{\frametitle{}}
% [containsverbatim] for using verbatim environment and \verb command. 
% [allowframebreaks] for automatic split of frames if the contents do not fit in a single slide. 
% [shrink] for shrinking the contents to fit in a single slide. 
% [squeeze] for squeezing vertical space.

\frame[plain]{\titlepage}


\frame{\frametitle{Plan}
 \setcounter{tocdepth}{1}
 \tableofcontents
}



\section{Introduction} % 


\frame{\frametitle{Présentation}

\begin{block}{Qu'est-ce que Seaside?}
 	Un framework écrit en Smalltalk pour le développement d'applications Web 
\end{block}


\begin{block}{Pourquoi est-ce intéressant pour vous?}
	\begin{itemize}
		\item Étudier une technologie alternative (autre que PHP, Java, Flash/Flex)
		\item Découvrir un framework \alert{innovant}
	\end{itemize}
\end{block}

}


\frame{\frametitle{Pré-requis}

	\begin{block}{Le développement Web}
		\begin{itemize}
			\item HTTP
			\item HTML, CSS
			\item Formulaires, POST, GET, ...
			\item Persistance 
		\end{itemize}
	\end{block}

	\begin{block}{Le langage Smalltalk}
		\begin{itemize}
			\item Langage orienté objets
			\item Dynamiquement typé
			\item Réflexif
		\end{itemize}
	\end{block}

}


\frame{\frametitle{Bibliographie (1)}

	\begin{block}{\url{http://www.seaside.st}}
		\centering\includegraphics[width=0.8\linewidth]{seaside-st.png}
	\end{block}
}

\frame{\frametitle{Bibliographie (2)}
\begin{block}{\url{http://book.seaside.st}}
	\centering\includegraphics[width=0.4\linewidth]{book-seaside-st}
\end{block}
}

\frame[plain]{

\vfill
\centering {\huge\color{mystructurecolor}Démo \code{WACounter}}
\vfill

}


\section{Rendu HTML et CSS} %


\begin{frame}\frametitle{Notion de \emph{composant}}
	
	\begin{block}{Qu'est-ce qu'un \emph{composant}?}
		\begin{itemize}
			\item un élément visuel d'une page HTML (généralement une \texttt{div})
			\item une instance d'une sous-classe de \code{WAComponent}
		\end{itemize}
	\end{block}
	
	\begin{block}{Notion de \emph{composant principal}}
		Toute application Web en Seaside possède un composant principal
	\end{block}
	
\end{frame}

\begin{frame}\frametitle{Générer du XHTML}

		\begin{block}{Affichage d'un composant en XHTML:}
	Un composant «sait s'auto-afficher» en HTML via sa méthode \code{renderContentOn:} (héritée de \texttt{WAComponent}).
	Cette méthode est automatiquement appelée par Seaside.
		\end{block}
		
	\begin{block}{La méthode \code{renderContentOn:}}
		prend un paramètre (nommé \texttt{html}) instance de \code{WARenderCanvas}
	\end{block}
	
	\begin{block}{La classe \code{WARenderCanvas}}
		fournit une interface permettant de facilement générer du XHTML \alert{valide} en utilisant des « pinceaux »
	\end{block}
	
\end{frame}


\begin{frame}[containsverbatim]\frametitle{Les « pinceaux » XHTML}
	\begin{block}{Le pinceau \texttt{paragraph}}
\begin{verbatim}
html paragraph: 'a simple text'.
html paragraph
   with: 'a simple text'
\end{verbatim}
	
	\end{block}
	
	\begin{block}{Le pinceau \texttt{div}}
\begin{verbatim}
html div: 'a simple text'.
html div
   with: 'a simple text'
\end{verbatim}	
	\end{block}

		\begin{block}{Utilisation combinée}
\begin{verbatim}
html div
   with: [html paragraph: [ 'a simple text' ].
      html image url: 'http://www.seaside.st/styles/logo-plain.png']
\end{verbatim}	
		\end{block}

\begin{block}{Le pinceau universel \code{render:}}
\begin{verbatim}
MyComponent>>renderContentOn: html
   html render: 'une chaine'.
   html render: anotherComponent
\end{verbatim}
\end{block}

\end{frame}


\begin{frame}[containsverbatim]\frametitle{En résumé}
	\begin{block}{}
		\begin{itemize}
			\item Un composant \code{MyApp} est une sous-classe de \code{WAComponent}
			\item Pour qu'un composant puisse être une application (point d'entrée), ajouter la méthode :
\begin{verbatim}
MyApp class>>canBeRoot
   ^true
\end{verbatim}				
			\item Pour que ce composant affiche du HTML, définissez la méthode:
\begin{verbatim}
MyApp>>renderContentOn: html
   html text: 'My App is so cool'
\end{verbatim}	
			\item Utiliser les pinceaux\\
			\url{http://book.seaside.st/book/fundamentals/rendering-components/learning-canvas-and-brush}
					
		\end{itemize}
	\end{block}
	
	\begin{alertblock}{A propos de \code{renderContentOn:}}
		\begin{itemize}
			\item Ne fait qu'afficher
			\item Pas d'instanciation, pas de modification, ...
		\end{itemize}
	\end{alertblock}
	
	
\end{frame}


\subsection{Gestion des styles CSS} %


\begin{frame}[containsverbatim]\frametitle{Gestion des classes de style}
	
	\begin{block}{Le message \code{class:} des pinceaux}
\begin{verbatim}
html div
   class: 'center';
   with: 'Seaside is cool'.
html paragraph
   class: 'highlight';
   with: 'Highlighted text'
\end{verbatim}
	\end{block}
	
		\begin{block}{Le message \code{class:if:} des pinceaux}
	\begin{verbatim}
html unorderedList with: [
   1 to: 10 do: [ :index |
      html listItem
         class: 'even' if: index even;
         with: index ] ]
	\end{verbatim}
	\end{block}
\end{frame}




\begin{frame}[containsverbatim]\frametitle{Gestion du code CSS}
	
	\begin{block}{Solution 1 : la méthode \code{style} des composants}
\begin{verbatim}
MyComponent>>style 
   ^ '.even { background-color: light-gray;}'
\end{verbatim}
	\end{block}
	
	\begin{block}{Solution 2 : un fichier CSS unique}
		\begin{enumerate}
			\item dans un fichier externe (\verb@<link ...@)
\begin{verbatim}
MyComponent>>updateRoot: anHtmlRoot
   super updateRoot: anHtmlRoot.
   anHtmlRoot stylesheet url: 'http://car.ensm-douai.fr/site21/mystyle.css'
\end{verbatim}
			\item dans une \texttt{FileLibrary} (sous classe de \code{WAFileLibrary})
\begin{verbatim}
MyFileLibrary addAllFilesIn: '/path/to/directory'
MyFileLibrary addFileAt: '/path/to/mystyle.css'

MyComponent>>updateRoot: anHtmlRoot
   super updateRoot: anHtmlRoot.
   anHtmlRoot stylesheet url: MyFileLibrary / #mystyleCss
\end{verbatim}
	
		\end{enumerate}
	\end{block}
\end{frame}


\begin{frame}[containsverbatim]\frametitle{Autre bibliothèques CSS}

\begin{center}
\Large
http://pharo.pharocloud.com/
\end{center}

\end{frame}

\section{Ancres et « callbacks »} %

\begin{frame}[containsverbatim]\frametitle{Le pinceau \texttt{anchor}}
	
	\begin{block}{Affichage d'un lien}
\begin{verbatim}
MyComponent>>renderContentOn: html
   html anchor
      url: 'http://www.seaside.st';
      with: 'Seaside Website'
\end{verbatim}
	\end{block}

\begin{block}{Notion de \texttt{callback}}
	Un block (fonction anonyme) qui sera executé \emph{automatiquement} lorsque l'ancre sera cliquée.
\alert{Pas d'affichage} (le renderer html est invalide) dans un callback.
\end{block}
	
\begin{block}{Exemple de \texttt{callback}}
\begin{verbatim}
MyComponent>>renderContentOn: html	
   html anchor
      callback: [ self clicked ] ;
      with: 'You already clicked me ', number, ' times'
MyComponent>>clicked
   number := number + 1
\end{verbatim}	
\end{block}

\end{frame}

\begin{frame}\frametitle{Exemples}
	\vfill
	\centering {\huge\color{mystructurecolor}Démo \code{WAMultiCounter}}
	\vfill
\end{frame}

\section{Gestion des formulaires} %

\begin{frame}[containsverbatim]\frametitle{Le pinceau \texttt{form}}
	
	\begin{block}{Edition des données d'un utilisateur}
\begin{verbatim}
UserEditorComponent>>renderContentOn: html
    html form: [
        html text: 'Login: '.
        html textInput
            callback: [ :value | self user login: value ];
            value: self login name.
        html break.
        html text: 'Email address: '.
        html textInput
            callback: [ :value | self user emailAddress: value ];
            value: self user emailAddress.
        html break.
        html submitButton
            callback: [ self save ];
            value: 'Save']
UserEditorComponent>>save
	self inform: self user login , '--' , self user emailAddress
\end{verbatim}
\end{block}
\end{frame}


\begin{frame}\frametitle{Gestion simplifiée des formulaires}
\vfill
\centering\url{http://book.seaside.st/book/fundamentals/forms/convenience}
\vfill
\begin{center}
	Une application exemple : Gestion d'une todo-list\\
	\url{http://book.seaside.st/book/in-action/todo}
\end{center}
\vfill
\end{frame}




% \section{Gestion de la session} % ch 18
	

\section{Composer des composants} %

\begin{frame}[containsverbatim]\frametitle{Le mécanisme \texttt{Call}/\texttt{Answer}}
\vfill
\centering Documentation : \url{http://book.seaside.st/book/components/calling}
\vfill
\end{frame}

% \subsection{Encapsulation} %


\section{Support Javascript} % 

\begin{frame}[containsverbatim]\frametitle{Scriptaculous}


\begin{block}{Scriptaculous}
\centering Documentation+démo : \url{http://scriptaculous.seasidehosting.st/}
\end{block}

\begin{block}{JQuery}
\centering Documentation+démo : \url{http://demo.seaside.st/javascript/jquery}
\end{block}

\begin{block}{JQueryUI}
\centering Documentation+démo : \url{http://demo.seaside.st/javascript/jquery-ui}
\end{block}

\end{frame}

% \section{Outils} % 
% 	
% \section{Le Debugueur} %
% 
% \subsection{Slime} %
% 
% \section{Persitence} % 

\section{Conclusion} % 

\frame{\frametitle{En résumé}

\begin{block}{Concepts et mécanismes innovants en Seaside :}
	
	% Session state is maintained on the server.
	% 	XHTML is generated completely in Smalltalk. There are no templates or “server pages” although it isn’t hard to build such things in Seaside.
	% 	You use callbacks for anchors and buttons, rather than loosely coupled page references and request IDs.
	% 	You use plain Smalltalk to define the flow of your application. You do not need a dedicated language or XML configuration files.
		
	\begin{itemize}
		\item un modèle de \emph{composants} et \emph{callbacks}
		\item du code HTML toujours valide
		\item les objets métiers sont toujours présents (paramètres GET/POST gérés automatiquement) 
		\item possibilité de débugguer facilement les applications Web 
		\item encapsulation de tehchnologies/frameworks Web 2.0: AJAX, JQuery, ...
	\end{itemize}
\end{block}

	
}

\frame[plain]{
\vfill
\centering {\huge\color{mystructurecolor}Alors?}
\vfill
\begin{center}
\includegraphics[width=0.6\linewidth]{BluePillRedPill}
\end{center}
}


\end{document}

